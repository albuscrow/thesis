% 学号
% !Mode:: "TeX:UTF-8"
% !TEX builder = LATEXMK
% !TEX program = xelatex
% 合作导师,如果没有合作导师,就在\documentclass选项栏中加上"nocpsupervisor"。
\documentclass[oneside,nocpsupervisor]{zjuthesis}

%\documentclass[doctor,twoside,nocpsupervisor]{zjuthesis}
\usepackage{color}
\usepackage{multirow}
\usepackage{tikz}
\usetikzlibrary{shapes,arrows,chains,positioning}
\usepackage[scientific-notation=true,round-precision=4,round-mode=figures]{siunitx}
\usepackage[justification=centering]{caption}

% 插图路径设置,图片放在figures 文件夹下。一般来说论文的插图比较多,通常按章节存
% 放,因此可以在以下命令中在按章节添加存放图片的文件夹路径。如以下这个路径中 ./
% 代表当前main.tex所在的目录,就是一般所说的当前文件夹;figures 文件夹就是子文件
% 夹,存放正文及附录中要用到的所有的图片,在figures 文件夹中的子文件夹就是存放各
% 个章节图片的文件夹,一般命名与相应章节的名字相同,如intro 章节用到的图片全放在
% 了intro 这个子文件夹下。
\graphicspath{%
	{./figures/intro/}%
	{./figures/sffd/}%
	{./figures/clip/}%
	{./figures/gpu/}%
	{./figures/mobile/}%
	{./figures/result/}%
	{./figures/implement/}%
}

% 论文中文标题
\title{改进的光滑自由变形及其应用}
% 论文英文标题
\englishtitle{Improved Smooth Free-Form Deformation and Its Applications}
% 作者,就是你的名字
\author{}
\enauthor{}
% 分类号
\classification{TM863}
% 单位代码
\serialnumber{10335}
% 密级
\secretlevel{公开}
% 学号
\studentnumber{}
% 指导教师
\supervisor{}
\ensupervisor{}
% 专业名称
\major{计算机技术}
\enmajor{Computer Technology}
% 研究方向
\research{图形学方向}
% 所在学院
\institute{计算机学院}
\eninstitute{College of Computer Science}
% 提交日期
\submitdate{2017年1月5日}
\ensubmitdate{Jan 5 2017}

% 中文题名页
\reviewerA{关羽\hspace{1.5em}五虎上将\hspace{1.5em}蜀汉}
\reviewerB{张飞\hspace{1.5em}五虎上将\hspace{1.5em}蜀汉}
\reviewerC{马超\hspace{1.5em}五虎上将\hspace{1.5em}蜀汉}
\chairperson{许攸\hspace{1.5em}文臣谋士\hspace{1.5em}曹魏}
\commissionerA{法正\hspace{1.5em}文臣谋士\hspace{1.5em}蜀汉}
\commissionerB{简雍\hspace{1.5em}文臣谋士\hspace{1.5em}蜀汉}
\commissionerC{麋竺\hspace{1.5em}文臣谋士\hspace{1.5em}蜀汉}
\defencedate{2017年3月5日}

\begin{document}
\maketitle
% \ZJUmakecover
% \ZJUmakeCNtitlepage
% \ZJUmakeENtitlepage
\absmatter
% !TEX root = ../main.tex

% 定义中文摘要和关键字
\begin{cabstract}
    自由变形\cite{Sederberg86}(Free Form Deformation,简称FFD)是一类编辑几何模型、生成柔体性动画的方法。由于其实现简单,与模型表示无关,所以被广泛的应用于计算机辅助设计与工程、计算机动画、医学影影等领域。另一方面,由于其较强的可拓展性,FFD\cite{Sederberg86}自提出以来就受到了学术届与工业界的广泛关注,该工作不仅实现了一种具体的编辑三维模型的方法,更重要的是提出了一套编辑三维模型的算法框架。

在该框架基础上,研究人员提出了许多方法以改进原有的FFD的一些不足之处,如变形结果走样,交互繁琐,效率低下等问题。

    崔的光滑自由变形\cite{Cui15}也是基于FFD框架的一种空间变形算法,主要在算法效率、变形结果质量这两方面对前人的工作作出了改进。使得用户可以实时的编辑三维模型,且能得到光滑细腻的变形。

    但其仍存在一些问题:1)算法在预处理阶段,会沿变形空间的结点盒对待变形模型形进行裁剪。但是裁剪可能会产生狭长多边形甚至退化多边形,这些多边形不仅会浪费计算资源,还会带来精度和程序鲁棒性的问题。2)光滑自由变形使用了CUDA做并行计算,以保证变形算法的实时性,但是CUDA只能在Nvidia的平台止运行,这大限制了光滑自由变形的通用性。

    因此,本文提出了一种基于光滑自由变形的新的变形算法,针对上述问题做出了相应的改进。

    首先,本文提出了一种新的多边形裁剪算法----三角均匀剖分算法。该算法以任意三角形为输入,对三角形进行分割,使产生的子三角形的形状尽可能接近正三角形且面积尽可能相等。此处,本文还对新旧两种分割算法进行了比较,并探索了新的分割算法对变形质量,运行效率的影响。

    其次,本文用OpenGL的Compute Shader实现了本文算法中除初始化之外的所有部分,在保证算法通用性的同时,尽可能地提高了算法的计算效率。同时,为了验证本文算法在不同平台下的通用性以及算法在计算资源较匮乏的情境下的表现,我们还在一个安卓平台的应用中实现了本文算法。

    最后,我们将本文算法与精确自由变形、光滑自由变形等方法进行多方面的对比。以体现本文方法的优势。
\end{cabstract}

\ckeywords{光滑自由变形,保留尖锐特征,OpenGL 计算着色器,图形处理单元,法向量场,三角形均匀剖分}

% !TEX root = ../main.tex

% 定义英文摘要和关键字

\begin{eabstract}
    Space deformation is one of key methods of shape manipulation and soft object animation in geometric modeling and computer graphics. Free-form Deformation (FFD) is the representative one among them. Due to its simplicity, intuitiveness, flexibility, it has been integrated into mainstream commercial animation softwares, such as 3DS Max, Maya, Softimage XSI, etc. In this thesis, an improved smooth FFD is proposed to address the problems of hardware-dependence and robustness in the traditional GPU-based FFD methods.

    Firstly, we propose a new method of uniformly subdividing the original model in the framework of smooth FFD. It receives a triangle as input and subdivides it to sub-triangles whose side length and area will be as equal as possible. The method can avoid pathological cases in traditional GPU-based FFD and generate the sub-triangles as uniform as possible. Furthermore, the subdivision method can be fully implemented on GPU. As a result, the smooth FFD becomes more robust and efficient.

    To overcome the problem of hardware-dependence in previous GPU-based FFD, we also proposed a new GPU-based improved smooth FFD by using OpenGL compute shader. All of the intensive computations are designed and implemented on GPU provided that the universality and efficiency of the proposed algorithm are guaranteed.  We also integrate the proposed algorithm into a 3D geometric modeling APP, namely, Wowtao, by the aid of the cross-platform property of OpenGL. It verifies the universality and feasibility of the proposed algorithm to achieve 3D shape editing in a resource-constrained hardware.

Finally, we compare the proposed method with the smooth FFD and accurate FFD thoroughly and show that the proposed method has the advantages of efficiency, robustness and cross-platform. It extends the potential application scenarios of FFD.
\end{eabstract}

\ekeywords{smooth FFD, sharp feature preserving, GPU, OpenGL compute shader, normal field}

\tocmatter
% 正文目录:
\tableofcontents
% 插图目录:
\listoffigures
% 表格目录:
\listoftables
%% !TEX root = ../main.tex
\begin{denotation}

\item[GPU] 图形处理器 (Graphics Processing Unit)
\item[FFD] 自由变形 (Free Form Deformation)
\item[GPGPU] 通用计算 (General-purpose computing on graphics processing units)
\item[VR] 虚拟现实 (Virtual Reality)
\item[AR] 增强现实 (Augmented Reality)
\item[CVT] Centroidal Voronoi tessellation

\end{denotation}

\mainmatter{}
% !TEX root = ../main.tex

\chapter{绪论}
%在FFD交互方式的探索中,研究者主要着眼于变形空间的选取。

    三维模型形状编辑是图形学及其相关领域中不可或缺的一部分,无论在研究还是工业领域都存在着广泛的需求和应用。在经历了空间变形、多分辨率变形、微分域变形等几个研究热点之后,这一领域已得到了较为成熟的发展。众多研究者提出了各种方法以满足不同的需求场景。

    但是随着网络带宽,计算机性能等硬件条件的改善,使得大众可以很方便的传输、显示体积较大的三维模型。另一方面,VR/AR的兴起,增强了用户对于模型本身的需求\footnote{这本质上是一种对于高承载密度的内容载体的需求。随着硬件环境的改善,用户对于内容的丰富程度的要求也随之提高,内容的载体也经历了从单纯的文字,到图片,再到音频/视频的发展。在可以预见的将来,三维图形将很快成为下一个流行的内容的载体,以满足用户越来越高的要求。}及创造三维内容的意愿。在这两方面的原因的推动下,用户对于三维模型编辑算法也提出了新的要求:如提高编辑后模型的质量以满足大众对于高质量模型的需求;改进交互方式以便用户快速的创作三维模型;提升算法的效率使得用户可以在硬件条件不变的情况下编辑创作更复杂的模型。

    为此,我们依然需要不断探索新的三维模型形状编辑算法,以满足当前环境下的新要求。

    空间变形作为一类简易,高效的三维模型形状编辑方法,近年来也发展出了一些新的变种以应对上文中提到的要求。比如:基于GPU的FFD算法\cite{chua2000, modat2010}借助GPU强大的并行计算能力加速算法,使得用户可以实时的用空间变形算法对三维模型进行编辑;精确自由变形\cite{Feng98, Feng00}从解析的角度处理因采样点不足引起的走样问题,从根本上解决了采样密度问题,并提升了变形结果质量,

    Cui等人的光滑自由变形\cite{Cui15}也是在这一环境下提出的一种空间变形算法。一方面,光滑自由变形用CUDA加速了算法中计算量较大的部分,使得用户可以实时地编辑模型。另一方面,算法在精确自由变形\cite{Feng98}的基础上引入了法向信息,在变形的最后阶段对变形结果进行调整,使得变形结果更加自然细腻。

    但是由于CUDA只能的NVIDIA的显卡上运行,所以Cui等人的这一工作也被限制在了NVIDIA平台上。另一方面,光滑自由变形沿用了精确自由变形\cite{Feng98, Feng00}的预处理方法。算法在预处理阶段,沿节点盒对模型的原始面片进行了切割。该步骤在精确自由变形中,是得到精确变形结果的必要条件。但在光滑自由变形中,变形结果是一个由拟合得到的降次的曲面片,所以在光滑自由变形中即使满足了沿节点盒切割这一条件,也无法得到精确的结果。反而可能因这一过程产生一些狭长的或者退化的三角形,这些三角形不仅会造成计算资源的浪费,还可能影响算法的稳定性。

    本文的工作基于光滑自由变形\cite{Cui15}。针对上述两个问题,进行了一系列深入的研究,并对其进行了一定的改进。使得算法更加鲁棒,高效、通用。

    另一方面,移动终端作为未来计算终端的发展趋势,正在逐渐替代传统桌面终端。人们在日常生活工作中也越来越多依赖于移动终端。我们可以预见将来会有越来越多的三维模型编辑方面的需求在移动终端产生。因此我们还将本文工作应用到了移动平台的APP中,借此对图形算法在移动平台上的应用进行了一定的探索。

    在介绍本文工作之前,让我们先来回顾一下前人的相关工作。


\section{相关工作}
    在诸多三维模型形状编辑算法中,空间变形是一类出现时间比较早,应用比较广泛的算法。该算法的基本思想最初由Barr\cite{Barr84}在1984年提出。Sederberg\cite{Sederberg86}于1986提出了经典的自由变形(Free-Form Deformation, 简称FFD),将这一思想完善为一个成熟的算法框架:
\begin{enumerate}
	\item 定义一个变形空间,也可以称之为中间体。
    \item 将待变形的模型“嵌入”到变形空间中,即计算模型上的采样点在变形空间中的参数坐标,该坐标在变形过程中保持不变。\label{item:2}
	\item 对变形空间进行变形。\label{item:3}
    \item 根据\ref{item:2}中得到的参数坐标与\ref{item:3}中得到的新变形空间,重新计算出采样点在欧氏空间中的坐标,从而达到变形的目的。
\end{enumerate}

\begin{figure}[htbp]
	\centering
	\begin{subfigure}[b]{.4\textwidth}
		\centering
		\includegraphics[width = \textwidth]{FFD_demo_0.png}
		\caption{原模型及中间体}\label{subfig:FFD_demo_0}
	\end{subfigure}
	\quad
	\begin{subfigure}[b]{.4\textwidth}
		\centering
		\includegraphics[width = \textwidth]{FFD_demo_1.png}
		\caption{FFD变形结果}\label{subfig:FFD_demo_1}
	\end{subfigure}
    \caption{自由变形示意图}\label{fig:FFD_demo}
\end{figure}

    该方法具有直观、简单、高效的优点,很多学者在在此基础上提出了大量关于FFD算法的研究与改进。

    变形空间的选取是空间变形算法的关键,算法的效率、交互方式、变形结果、适用的变形需求等会因中间体表达方式的改变而改变。所以很多研究工作都着眼于中间体的改进:

    Griessmair\cite{Griessmair89}和Lamousin\cite{lamousin1994}分别用B样条体和非均匀有理B样条体替换传统FFD\cite{Sederberg86}中的Bézier体,由于B样条体和非均匀有理B样条体的局部支撑性,使得模型的变形具有局部可调性。Coquillart\cite{coquillart1990}则通过移动、“焊接”长方体控制网格中的部分控制顶点,使得控制网格不再局限于长方体,而是将其转变圆柱等更加复杂的形状。这一改变虽然增加了变形模型嵌入变形空间过程中的计算量且对于控制网格的形状有一定的限制,但是改进了交互方式,使用户能够进行更为复杂的变形。

    后续的学者进一步提出了以曲面、曲线甚至是以点为变形工具来构建变形空间。这些方法各自适用于不同的变形场合。如以曲线构建变形空间的FFD适用于基于骨架变形或者细长物体的变形,基于曲面的变形比较适用于形状较为扁平的模型的变形。总体而言,用户编辑模型时可以控制的变量越多,算法对模型的编辑能力就越强,能完成更加复杂的变形,但同时用户的交互也越复杂。在Gain和Beckmann的综述\cite{Gain08}中,作者对常见的FFD按照选用变形空间的不同进行了分类,并从局部可调性、交互友好度、时空效率、是否保拓扑等方面详细对比了各个方法的优劣。

    上述这些方法中,无论中间体如何选择,用户都需要通过操作控制顶点来对中间体进行变形,中间体再通过采样点的参数坐标将变形“传导”到待变形的模型中。这一交互方式对普通用户而言并不友好,控制顶点的位移与待变形模型的顶点的位移并不完全一致。用户需对中间体的数学表示有所了解,才能实现更为复杂的变形意图。

    为了解决这一问题,Hsu\cite{hsu1992}最先提出了直接自由变形,这一算法允许用户指定待变形模型上的一组点,并由用户提供这组点变形前后在欧氏空间中的位移。以前述信息为输入,算法将自动求出一组合适的控制顶点,使得用户指定的点的在变形过程中的位移与输入保持一致。最后由这组控制顶点求得最终变形结果。

    随后,其它研究者也在此基础上提出了许多类似方法,统称为直接自由变形。这一类方法直接关联用户输入和变形结果,使用户直接编辑模型变形后关键点的位置,即可等到整体的变形结果,而无需关心中间体的数学表示。使得FFD的交互更加便捷、直观,大幅度提高了FFD算法的易用性。

    传统FFD方法的另一个不足之处是变形过程中的走样问题。

    由于传统FFd方法的变形是作用到待编辑模型的采样点上的,再由采样点变形后的位置,还原出模型的变形结果。所以,最终得到的变形结果的质量很大程度上依赖于采样点的密度。\autoref{fig:sample_problem}很好的演示了这一问题。柜子的四个腿在原模型中由狭长的三角形组成,并且从视觉上看是与底部十字木条连在一起的。但是若以模型顶点为采样点,进行如\autoref{subfig:sample_problem_1}所示的变形后,柜子腿从视觉上与底部的十字木条分开了。这是由于采样点过于稀疏,变形只作用于柜子腿底部的顶点,而未作用于柜子腿与十字木条相连的部分造成的。

\begin{figure}[htbp]
	\centering
	\begin{subfigure}[b]{.4\textwidth}
		\centering
		\includegraphics[width = \textwidth]{sample_problem_1.png}
		\caption{原模型}\label{subfig:sample_problem_0}
	\end{subfigure}
	\quad
	\begin{subfigure}[b]{.4\textwidth}
		\centering
		\includegraphics[width = \textwidth]{sample_problem_2.png}
		\caption{传统FFD变形结果}\label{subfig:sample_problem_1}
	\end{subfigure}
    \caption{传统FFD由于采样密度不足造成的精度问题}\label{fig:sample_problem}
\end{figure}

    解决这一问题最直接的思路是通过均匀加密采样增加采样点的密度,但这一方法会造成性能上较大的开销。更进一步的方法\cite{parry1986, gain1999}是根据面片大小和曲面的曲率,自适应确定采样密度.自适应采样虽然从性能上较之均匀加密采样有了一定的提升,但自适应算法实现相对复杂,并且无法很好的处理一些奇异情况。

    为了更好的解决FFD中的走样问题,Feng\cite{Feng98}提出了“精确自由变形”。从一个较高的抽象层次观察传统自由变形和精确自由变形,前者是以采样点为变形对象的,而后者是以组成模型的面片为变形对象的。精确自由变形算法的直接输出也不再是采样点变形后的位置,而是初始三角面片经由变形得到的新的曲面片。这样一来就避免了传统方法中采样过程所带来的走样问题。

    Feng在其工作中首先证明了:若自由变形采用的中间体为B样条体,且待变形的三角面片位于唯一的节点盒之内\footnote{即三角面片只位于某个节点盒之内,而没有跨越节点盒},那么该三角面片的精确的变形结果为一个三角Bézier曲面片,且其次数为作为变形空间的B样条体三个维度上的次数之和。然后,作者再通过函数复合\cite{derose1988, derose1993}和位移算子\cite{chang1984},计算出三角Bézier曲面片的控制顶点,从而得到精确而自然的变形结果。\autoref{fig:sample_problem_affd}中对比了精确FFD与传统FFD的变形结果。


\begin{figure}[htbp]
	\centering
	\begin{subfigure}[b]{.3\textwidth}
		\centering
		\includegraphics[width = \textwidth]{sample_problem_1.png}
		\caption{原模型}
	\end{subfigure}
	\quad
	\begin{subfigure}[b]{.3\textwidth}
		\centering
		\includegraphics[width = \textwidth]{sample_problem_2.png}
		\caption{传统FFD变形结果}
	\end{subfigure}
	\quad
	\begin{subfigure}[b]{.3\textwidth}
		\centering
		\includegraphics[width = \textwidth]{sample_problem_affd.png}
		\caption{精确FFD变形结果}
	\end{subfigure}
    \caption{传统FFD和精确FFD的结果对比}\label{fig:sample_problem_affd}
\end{figure}

作者在后续工作\cite{Feng00, Feng02}中对上述算法进行了改进,不仅使算法开销更低,还提升了算法的通用性。尽管如此,采用精确自由变形编辑较大的模型时,仍然无法做到实时交互。

另一方面,通用计算在近年来得到了较大的发展,GPU从一个仅负责图形绘制的专用硬件渐渐演变为一个多线程、高带宽的通用计算硬件。CUDA、OpenCL等通用计算框架的出现,更是进一步方便了应用程序利用GPU的强大的计算能力,以大幅提升其运行速度。FFD的构架下的算法大多是计算密集型的,且这些算法能较好的适用于单指令流多数据流计算模型,所以随着通用计算的兴起,自然有很多学者利用GPU加速各类FFD,从而使这些算法能对大型三维模型进行实时编辑。

其中,Cui\cite{Cui13}的工作成功地将CUDA应用到了精确自由变形中,采用GPU实现的算法相较于原先的CPU版本快了50倍左右。另一方面,无论是传统的FFD还是精确FFD,在其变形过程中都只考虑了待变形的模型的几何信息,而未考虑模型的法向信息。所以变形之后得到的模型会有不光滑的走样,\autoref{subfig:saffd_0}和\autoref{subfig:saffd_1}演示了这种走样现象。为了解决这些问题,Cui在其后续工作\cite{Cui15}中提出了光滑自由变形。在精确自由变形的基础上,光滑自由变形根据法向信息对变形后得到的曲面片进行了微调,以得到如\autoref{subfig:saffd_2}所示的光滑外观。

\begin{figure}[htbp]
	\centering
	\begin{subfigure}[b]{.3\textwidth}
		\centering
		\includegraphics[width = \textwidth]{saffd_0.png}
		\caption{传统自由变形结果}\label{subfig:saffd_0}
	\end{subfigure}
	\quad
	\begin{subfigure}[b]{.3\textwidth}
		\centering
		\includegraphics[width = \textwidth]{saffd_1.png}
		\caption{精确自由变形结果}\label{subfig:saffd_1}
	\end{subfigure}
	\quad
	\begin{subfigure}[b]{.3\textwidth}
		\centering
		\includegraphics[width = \textwidth]{saffd_2.png}
		\caption{光滑自由变形结果}\label{subfig:saffd_2}
	\end{subfigure}
    \caption{传统自由变形、精确自由变形和光滑自由变形的结果对比}\label{fig:sample_problem_saffd}
\end{figure}

移动终端经过近几年的高速发展,正逐渐成为人们主要的个人计算设备,人们正将越来越多的计算任务从传统的桌面端迁移到移动终端。为了应对这一趋势,Hong\cite{hong2013}尝试了将FFD算法实现在移动设备上,并取得了不错的结果。


\section{本文内容安排}
本文的内容安排如下:
第一章介绍了本文研究背景,回顾了三维模型编辑领域的相关工作。其中,重点介绍了自由变形的发展和局限。同时引出了本文希望解决的问题。

本文基于光滑自由变形,在此基础上对其中的“切割原始三角形”这一步骤进行了改进,并用OpenGL Compute实现了所有算法步骤,包括“切割原始三角形”\footnote{这一步骤在光滑自由变形中是在CPU上实现的}。除了以上两点改进。本文算法均与光滑自由变形一致。所以本文在第二章先对光滑自由变形\cite{Cui15}的各个步骤进行了简单介绍。

第三章详细介绍了本文提出的新的三角形均匀剖分算法。参数$l$是本文分割方法的重要参数,用以控制子三角形的分割粒度。所以本章还分析了参数$l$对算法效率、生成的子三角形的质量、变形结果的精度的影响。

第四章介绍了算法的具体实现。算法将所有计算过程实现成了若干个OpenGL的Compute Shader,以利用GPU的并行能力提高算法的运行速度,使用户在编辑三维模型形状时能实时交互。新的三角形分割算法本身难以通过目前的GPU通用计算架构实现,所以我们提前计算不同的三角形的分割方案并储存起来,GPU在分割三角形时直接通过三边比例找到合适的分割方案进行分割。该方案使整个变形过程都在GPU中执行,能够大幅提高侵害效率。同时,OpenGL作为跨平台的图形编程接口,还给本文方法带来了良好的通用性。

第五章讲述了本文算法在移动平台上的应用。我们实现了一个安卓平台上的模拟陶瓷制作过程的应用----哇陶\footnote{该应用是一个陶瓷制作过程模拟仿真软件,可以让用户制作出个性化的瓷器}。本文算法应用在泥胚塑形环节,以帮助用户得到满意的造型。

第六章对本文算法与光滑自由变形\cite{Cui15},精确自由变形\cite{Feng00}进行了多方面的对比,包括绘制效果、算法时间、变形精度等。

最后一章对本文工作进行了总结,概括了本文方法的优势与适用场景。还提出了本文方法的一些不足之处并对未来的研究工作进行了展望。






% !TEX root = ../main.tex

\chapter{三角贝赛尔曲面片表示的光滑自由变形概述}
本文方法作为光滑自由变形\cite{Cui15}的改进算法,依旧遵循经典自由变形的算法框架。下文将沿着该框架的流程,对光滑自由变形算法进行简单介绍。

\section{定义变形空间}
精确自由变形选用B样条体作为变形空间,记作$\mathbf R(u,v,w)$:
\begin{equation}
	\footnotesize
	{\mathbf R}(u,v,w) 
	= \sum_{i=0}^{m_u-1}\sum_{j=0}^{m_v-1}\sum_{k=0}^{m_w-1} {\mathbf
	R}_{ijk}N_{i,n_u}(u)N_{j,n_v}(v)N_{k,n_w}(w)
	\label{equ:Ruvw}
\end{equation}
其中,$n_u$、$n_v$、$n_w$与$m_u$、$m_v$、$m_w$分别表示B样条体三个维度上的次数与控制顶点个数。$\{\mathbf R_{ijk}\}_{i=0,\hspace{6 pt} j=0,\hspace{8 pt} k=0}^{m_u-1,m_v-1,m_w-1}$表示$m_u\times m_v\times m_w$个控制顶点。$\{N_{i,n_u}(u)\}_{i=0}^{m_u-1}$, $\{N_{j,n_v}(v)\}_{j=0}^{m_v-1}$ 和 $\{N_{k,n_w}(w)\}_{k=0}^{m_w-1}$是B样条基函数。

此B样条体三个维度上的节点向量分别是$\{u_i\}^{n_u+m_u}_{i=0}$, $\{v_i\}^{n_v+m_v}_{j=0}$ 和 $\{w_k\}^{n_k+m_k}_{k=0}$。由此定义的三维区域$[u_i, u_{i+1}] \times [v_j, v_{j+1}] \times [w_k, w_{k+1}]$我们称之为节点盒,其中$n_u\leq i < m_u$,$n_v\leq j < m_v$,$n_w\leq k < m_w$。

\subsection{沿节点盒切割多边形面片}
根据论文\cite{Feng98, Feng00},完全在某个节点盒内的三角面片,其变形后的结果是一个三角贝赛尔曲面片,记作${\mathbf P}(u,v,w)$:\label{section:split}

\begin{equation}
	\footnotesize
	{\mathbf P}(u,v,w)
	= \sum_{\substack{i+j+k=n \\ 0\leq i,j,k\leq n}} {\mathbf P}_{ijk}B^n_{ijk}(u,v,w), \hspace{8 pt} u,v,w\ge0,
		\hspace{8 pt}u+v+w=1
	\label{equ:Puvw}
\end{equation}

其中$\{B_{ijk}^n(u,v,w)=\frac{n!}{i!j!k!}u^iv^jw^k \mid i+j+k=n\}$是定义在一个三角形上的伯恩斯坦基函数,$n(=n_u+n_v+n_w)$为曲面片的次数,$\{\mathbf P_{ijk} \mid i+j+k=n\}$为些三角贝赛尔曲面片的$m(=(n+1)(n+2)/2)$个控制顶点。

所以由平面多边形构成的模型在嵌入B样条体之后,需要沿上述节点盒切割,且得到的非三角形的子多边形还需再进行三角化。以保证变形后的结果可以用三角贝赛尔曲面片表示。

\section{模型嵌入变形空间}
“嵌入”过程,其实是计算“待变形的对象”在变形空间中的参数坐标的过程。从用户角度来看,“侍变形对角”就是待变形的模型,但是从算法实现的角度来看,真正参与变形的实际上是模型上的采样点。因此,“嵌入”就是通过嵌入函数$U=E(X)$将采样点从笛卡尔坐标系映射到变形中间中的过程。其中$X$为采样点在笛卡尔坐标系中的坐标,$U$为采样点在变形空间的参数坐标。

所以嵌入变形空间这一过程有两个要点:确定变形函数、选取采样点。

\subsection{确定变形函数}

嵌入函数$E(X)$由变形空间决定。光滑自由变形通过\cite{Feng02}中的方法构造B样条体,使得其具有如下性质:嵌入其中的点的参数坐标与其笛卡尔坐标系的坐标相等。即$E(X)=X$。

\subsection{采样点的选取}
不同的种类的FFD算法,需要选取不同的采样点。

传统自由变形以待变形模型的顶点为采样点,这将导致如图\autoref{fig:sample_problem}所示的走样问题。均匀加密采样只能在一定程度上解决走样问题,且随着采样点密度的增加,无论是时间还是空间上的开销均会显著增加。自适应的加密采样是在加密采样思路下解决走样问题的更进一步的尝试。相对均匀加密采样而言,其虽然能减少计算量,但是实现复杂,无法很好的应对一些特殊情况。以上两种方法匀无法人根本上解决走样问题。

精确自由变形\cite{Feng00}通过另一种思路从根本上解决了走样问题。如\ref{section:split}所述,三角面片变形后是一个三角贝赛尔曲面片,所以,只要用FFD求得三角面片上的$m$个均匀采样点在变形之后的位置,就可以通过多项式插值高效计算出三角贝赛尔曲面片的控制顶点。这一方法相对于加密采样的优势在于其结果是一个解析表达的贝赛尔曲面片,可以根据需求细化绘制出不同精度的变形结果,而无需改变采样点个数。

光滑自由变形\cite{Cui15}进一步优化了精确自由变形的计算量。首先,Cui指出作为变形结果的三角贝赛尔曲面片,其次数往往高于三次,高次的结果不仅会直接产生高昂的计算代价,还会间接的增加算法其它部分的复杂度。然后,Cui通过实验发现,三次的三角贝赛尔曲面片能提供足够的灵活性以拟合精确的自由变形结果。所以,Cui选用了三次贝赛尔曲面片来表示变形结果。该曲面片由带约束的拟合方法求出,该方法需要$3\times3$个约束点,$m$个拟合点。所以,光滑自由变形共需要$9+m$个采样点。\autoref{fig:saffd_sample_point}中展示了当$m=4$时,采样点的选取方式,黄色点为约束点,蓝色点为拟合点。

\begin{figure}[htbp]
	\centering
	\includegraphics[width = 0.5\linewidth]{saffd_sample_point.png}
	\caption{三角面片上的拟合点的约束点示意图}\label{fig:saffd_sample_point}
\end{figure}


\section{变形}
如前文所述,实际参与变形的是控制顶点。通过上述步骤,我们已经将采样点嵌入到B样条体中。当用户编辑B样条体控制顶点的位置后,算法先将采样点的参数坐标和控制顶点位置代入\autoref{equ:Ruvw},就可以求得采样点新的位置。再以这些采样点新的位置为输入,通过带约束的拟合方法,求出作为变形结果的三角贝赛尔曲面的控制顶点。

为了得到视觉上更加细腻的变形结果,算法还将根据法向信息对控制顶点进行微调。最终输出光滑并保持尖锐特征的结果。

以上就是光滑自由变形的大致步骤,下面将介绍本文算法在此基础上进行的改进。

% !TEX root = ../main.tex

\chapter{三角均匀剖分算法}
如上文所述,光滑自由变形需将模型沿节点盒切割,并将非三角形的面片三角化。这一过程很可能产生狭长三角形或者蜕化三角形,如\autoref{subfig:clip_compare0}所示,颜色越红表示三角形质量越差。过多的此类三角形不仅会浪费计算资源,还可能带来其它数值计算方面的问题。

另一方面,沿节点盒切割只是为了保证变形结果是严格的$n$次三角贝赛尔曲面片。若略去这一步骤,算法仍能继续,只不过结果会变得不精确。考虑到在光滑自由变形中,作者采用以精度换取效率的策略,用三次的三角贝赛尔曲面来拟合n次的精确结果,也就是说光滑自由变形的结果只是精确结果的近似。因此,光滑自由变形中“沿节点盒切割以保证结果精确”这一步骤就显的可有可无了。

而且,我们通过更进一步的观察发现,在光滑自由变形中,跨多个节点盒的三角面片只要足够小,其变形后在精度上的误差,相对于在节点盒内的三角面片而言,并不会显著增加。

因此,本文尝试提出一种更好的三角形分割算法,以替换光滑自由变形中的“沿节点盒切割”这一步骤。新算法不再沿节点盒切割三角面片,而是将三角面片按以下两个要求分割成子三角形:
\begin{itemize}
    \item 所有子三角形的三边长都尽可能相等。以得到形状尽可能接近正三角形,且面积尽可能相同的子三角形。
    \item 所有子三角形的顶点均不可位于其它子三角形的边上。以避免变形后产生裂缝。
\end{itemize}

我们的新算法相比原来的沿节点盒分割的算法而言有两点优势:
\begin{itemize}
        \item 切分出来的子三角形尽可能“正”,不会产生新的狭长或蜕化的三角形。
        \item 参数$l$使得用户可以控制产生的子三角形的大小。
\end{itemize}


\section{算法实现}\label{clip_algorithm}
首先我们先定义一些符号以方便描述算法实现。$l$,$t$为算法的输入,$l$表示子三角形边长的期望值,算法分割产生的子三角形的三边的长度需尽可能接近$l$。$t$表示待分割的三角面片。$\{e_i\}^{2}_{i=0}$为$t$的三条边。$\{len_i\}^{2}_{i=0}$代表三边的边长。$t$的每条边会被均匀分成$\lceil len_i/l \rceil$段。$p_{ij}$表示三角形第i条边的第j个分割点。

算法流程:
\begin{enumerate}
    \item 找出三角面片最小的内角$\alpha$,不妨假定角$\alpha$的两条边为$e_0$, $e_1$。
    \item 将$e_0$、$e_1$分别均匀分成$\lceil len_0/l \rceil$、$\lceil len_0/l \rceil$段,产生的切割点的有序集合\footnote{切割点包括边的首尾端点}为$\{p_{0j}\}^{\lceil len_0/l \rceil}_{j=0}$、$\{p_{1j}\}^{\lceil len_1/l \rceil}_{j=0}$,且$j$沿角$\alpha$的顶点至另一端点方向依次增长。分割点如\autoref{subfig:clip1}所示。
    \item 将子三角形$p_{00}p_{01}p_{11}$分割下来,如\autoref{subfig:clip2}所示。
    \item 剩下的部分是一个类似于梯形的四边形,如\autoref{subfig:clip2}中红色部分所示。如果$\lceil len_0/l \rceil == \lceil len_1/l \rceil$,我们依次将四边形$\{p_{0j}p_{1j}p_{1(j+1)}p_{0(j+1)}\}^{\lceil len_0/l \rceil - 1}_{j=0}$分割成子三角形,如\autoref{subfig:clip6}、\autoref{subfig:clip9}所示,直到将所有四边形均分割完毕,然后直接跳转到步骤\ref{CVT};否则\footnote{$\lceil len_0/l \rceil \ne \lceil len_1/l \rceil$},我们依次将四边形$\{p_{0j}p_{1j}p_{1(j+1)}p_{0(j+1)}\}^{min(\lceil len_0/l \rceil, \lceil len_1/l \rceil) - 2}_{j=0}$分割成子三角形,剩余部分如\autoref{subfig:clip16}所示。\label{recursion}
    \item \autoref{subfig:clip16}中剩余部分若逆时针旋转90度,我们可以发现剩余部分和\autoref{subfig:clip3}类似。所以我们递归的进行步骤\ref{recursion},直到三角形分割完毕。\label{recursion2}
    \item 对分割结果进行5次CVT优化,使子三角形边长更加接近$l$。\label{CVT}
\end{enumerate}

在步骤\ref{recursion2}中,递归到最后阶段,可能会出现步骤\ref{recursion}处理不了的特殊情况。但是特殊情况的类型不会很多,所以我们为每一种特殊情况指定了对应的分割方案。

\begin{figure}[htbp]
	\centering
	\begin{subfigure}[b]{.32\textwidth}
		\centering
		\includegraphics[width = \textwidth]{clip_figure0.png}
		\caption{初始三角面片}\label{subfig:clip0}
	\end{subfigure}
	\begin{subfigure}[b]{.32\textwidth}
		\centering
		\includegraphics[width = \textwidth]{clip_figure1.png}
		\caption{初始三角面片}\label{subfig:clip1}
	\end{subfigure}
	\begin{subfigure}[b]{.32\textwidth}
		\centering
		\includegraphics[width = \textwidth]{clip_figure2.png}
		\caption{分割最小角$\alpha$的两条边}\label{subfig:clip2}
	\end{subfigure}

	\begin{subfigure}[b]{.32\textwidth}
		\centering
		\includegraphics[width = \textwidth]{clip_figure3.png}
		\caption{切割第一个子三角形}\label{subfig:clip3}
	\end{subfigure}
	\begin{subfigure}[b]{.32\textwidth}
		\centering
		\includegraphics[width = \textwidth]{clip_figure4.png}
		\caption{第一个待分割层}\label{subfig:clip4}
	\end{subfigure}
	\begin{subfigure}[b]{.32\textwidth}
		\centering
		\includegraphics[width = \textwidth]{clip_figure5.png}
		\caption{分割待分割层的上下底边}\label{subfig:clip5}
	\end{subfigure}

	\begin{subfigure}[b]{.32\textwidth}
		\centering
		\includegraphics[width = \textwidth]{clip_figure6.png}
		\caption{分割待侵害层的上下底边}\label{subfig:clip6}
	\end{subfigure}
	\begin{subfigure}[b]{.32\textwidth}
		\centering
		\includegraphics[width = \textwidth]{clip_figure9.png}
		\caption{分割待侵害层的上下底边}\label{subfig:clip9}
	\end{subfigure}
	\begin{subfigure}[b]{.32\textwidth}
		\centering
		\includegraphics[width = \textwidth]{clip_figure16.png}
		\caption{继续侵害红色部分}\label{subfig:clip16}
	\end{subfigure}

	\begin{subfigure}[b]{.32\textwidth}
		\centering
		\includegraphics[width = \textwidth]{clip_figure26.png}
		\caption{对特殊情况进行处理}\label{subfig:clip26}
	\end{subfigure}
	\begin{subfigure}[b]{.32\textwidth}
		\centering
		\includegraphics[width = \textwidth]{clip_figure33.png}
		\caption{分割结果}\label{subfig:clip33}
	\end{subfigure}
	\begin{subfigure}[b]{.32\textwidth}
		\centering
		\includegraphics[width = \textwidth]{clip_figure34.png}
		\caption{对结果进行CVT优化}\label{subfig:clip34}
	\end{subfigure}
	\caption{均匀三角剖分算法}\label{fig:clip}
\end{figure}

CVT\cite{du1999}优化可以使分割更加均匀。且该过程可以迭代进行,使分割结果越来越均匀。\autoref{fig:CVT}展示了迭代过程,红色是优化之后的结果,灰色的是优化之前的。我们可以发现,第5次及之后的迭代收益不大。所以我们最终对结果进行5次CVT迭代优化。

\begin{figure}[htbp]
	\centering
	\begin{subfigure}[b]{.49\textwidth}
		\centering
		\includegraphics[width = \textwidth]{cvt_for_paper0.png}
		\caption{第1次CVT优化}
	\end{subfigure}
	\begin{subfigure}[b]{.49\textwidth}
		\centering
		\includegraphics[width = \textwidth]{cvt_for_paper1.png}
		\caption{第2次CVT优化}
	\end{subfigure}

	\begin{subfigure}[b]{.49\textwidth}
		\centering
		\includegraphics[width = \textwidth]{cvt_for_paper2.png}
		\caption{第3次CVT优化}
	\end{subfigure}
	\begin{subfigure}[b]{.49\textwidth}
		\centering
		\includegraphics[width = \textwidth]{cvt_for_paper3.png}
		\caption{第4次CVT优化}
	\end{subfigure}

	\begin{subfigure}[b]{.49\textwidth}
		\centering
		\includegraphics[width = \textwidth]{cvt_for_paper4.png}
		\caption{第5次CVT优化}
	\end{subfigure}
	\begin{subfigure}[b]{.49\textwidth}
		\centering
		\includegraphics[width = \textwidth]{cvt_for_paper5.png}
		\caption{1~10次CVT优化}
	\end{subfigure}
    \caption{CVT优化结果} \label{fig:CVT}
\end{figure}

\section{关于$l$的讨论}
前文提到的$l$是一个很关键的参数,$l$不仅会影响算法效率、子三角形的质量\footnote{这里的质量表示接近正三角形的程度}、还能影响变形结果的精度。但是,$l$并不是影响这些的直接因素,对子三角形的质量和算法效率产生直接影响的是子三角形的数目,而对变形结果的精度产生直接影响的是子三角形的平均面积。$l$则通过改变子三角形数目或平均面积,间接地对算法效率、子三角形的质量、变形结果的精度产生影响。

定性的来说$l$越小,子三角形数目越多,面积越小,同时子三角形的质量越高,变形结果的误差越小,但算法的计算代价越高。为了进一步明确$l$对算法的影响,我们进行了一系列实验以定量的分析$l$。

实验选用了立方体模型,有12个面片,且模型被归一化到$[-1, 1]^3$。之所以选择由大三角面片组成的模型,是为了让三角面片随$l$变化切割出不同大小的子三角形。\autoref{fig:l-number}显示了$l$与子三角形数量的关系,蓝色表示以立方体为模型进行切割,立方体原始的三角面片平均面积为2.0,绿色表示以兔子玩偶为模型进行切割,兔子玩偶原始的三角面片平均面积为0.0007。从中我们可以发现兔子玩偶模型原始三角片面过小,当$l$大于某个值的时候,其不再被切割,子三角形的数目也保持不变,这样不利于我们探索$l$在对变形算法产生的影响。而立方体由于原始三角面片较大,能随$l$变大切割出不同数目的子三角形。可以帮助我们更好的研究$l$对算法的影响。

变形空间的次数是$3\times3\times3$,控制顶点个数是$5\times5\times5$。实验的自变量是$l$,由$0.1$增长到$2\sqrt{3}$\footnote{$2\sqrt{3}$为变形空间对角线长度}。因变量是算法效率、子三角形的质量、变形结果的精度。以下是实验结果。

%兔子模型的平均三角形面积为总表面积/三角形数目
%5.880918 / 8400 = 0.0007001092857142858
\begin{figure}[htbp]
	\centering
	\includegraphics[width = 0.8\textwidth]{l-number.png}
	\caption{$l$取值-子三角形数目关系图}\label{fig:l-number}
\end{figure}

\subsection{算法效率}
从\autoref{fig:l-number}可以看出不同的$l$可能产生相同的分割结果,所以\autoref{fig:l-time0}中,$l$与算法运算时间的关系不是很直观。如前文所述,$l$通过影响子三角形数目,从而影响算法运算时间,所以我们直接探究子三角形数目和算法运算时间的关系。如\autoref{fig:l-time1}中所示,大体上来看,算法运行时间与子三角形数目呈线性正相关。

\begin{figure}[htbp]
	\centering
	\begin{subfigure}[b]{.45\textwidth}
	    \centering
	    \includegraphics[width =\textwidth]{l-time0.png}
	    \caption{$l$取值-算法运行时间关系图}\label{fig:l-time0}
	\end{subfigure}
	\begin{subfigure}[b]{.45\textwidth}
	    \centering
	    \includegraphics[width =\textwidth]{l-time1.png}
	    \caption{三角形数目-算法运行时间关系图}\label{fig:l-time1}
	\end{subfigure}
	\caption{$l$取值-算法效率}\label{fig:l-time}
\end{figure}

\subsection{子三角形质量}
有很多指标都可以衡量三角形质量\cite{pebay2003},我们选取了其中计算量较少的一个$q(t)=r_t/max(\{e_i\}^{2}_{i=0})$,即三角形外接圆的半径除以其最长边。该指标的取值范围是$(0, \frac{\sqrt{3}}{2}]$,当三角形$t$为正三角形时,$q(t)$取到最大值$\frac{\sqrt{3}}{2}$;三角形$t$越狭长,$q(t)$的值越接近0。为了更直观的用该指标表示三角形质量,我们修改将$q(t)$修改为$q(t)=r_t/max(\{e_i\}^{2}_{i=0})*2\sqrt{3}$,使其取值范围为$(0, 1]$。\autoref{fig:triangle_quality_compare}可视化了两种三角形分割方法产生的三角形的质量,可见本文方法产生的三角形质量要好很多。

\begin{figure}[htbp]
	\centering
	\begin{subfigure}[b]{.45\textwidth}
		\centering
		\includegraphics[width = \textwidth]{clip_compare0.png}
		\caption{沿节点盒切割}\label{subfig:clip_compare0}
	\end{subfigure}%
	\begin{subfigure}[b]{.45\textwidth}
		\centering
		\includegraphics[width = \textwidth]{clip_compare1.png}
		\caption{本文方法}\label{subfig:clip_compare1}
	\end{subfigure}
	\caption{两种切割算法产生的子三角形的质量对比}\label{fig:triangle_quality_compare}
\end{figure}

对于不同的$l$对三角形质量的影响如\autoref{fig:l-quality}所示。我们分别作了$l$取值、子三角形数目与子三角形质量的关系图。可见$l$越小,三角形质量越高。但是由于切割算法中存在取整操作,所以\autoref{fig:l-quality}存在许多跳变的地方。

\begin{figure}[htbp]
	\centering
	\begin{subfigure}[b]{.45\textwidth}
		\centering
		\includegraphics[width = \textwidth]{l-quality0.png}
		\caption{$l$取值-子三角形质量关系图}\label{subfig:l-quality0}
	\end{subfigure}%
	\begin{subfigure}[b]{.45\textwidth}
		\centering
		\includegraphics[width = \textwidth]{l-quality1.png}
		\caption{子三角形数目-子三角形质量关系图}\label{subfig:l-quality1}
	\end{subfigure}
	\caption{$l$取值-子三角形的质量对比}\label{fig:l-quality}
\end{figure}

\subsection{变形结果精度}
相比于光滑自由变形,本文方法会产生跨节点盒的子三角形。这些三角形会引入额外的误差。但是只要子三角足够小,误差就能控制在可接受范围内。我们同样做了实验以验证$l$取值与变形误差的关系。从\autoref{fig:l-error}中可以看出,变形结果的几何误差基本与子三角形面积呈线性正相关。
\begin{figure}[htbp]
	\centering
	\begin{subfigure}[b]{.45\textwidth}
		\centering
		\includegraphics[width = \textwidth]{l-error0.png}
		\caption{$l$取值-子三角形质量关系图}\label{subfig:l-error0}
	\end{subfigure}%
	\begin{subfigure}[b]{.45\textwidth}
		\centering
		\includegraphics[width = \textwidth]{l-error1.png}
		\caption{子三角形数目-子三角形质量关系图}\label{subfig:l-error1}
	\end{subfigure}
	\caption{$l$取值-变形误差}\label{fig:l-error}
\end{figure}

% !TEX root = ../main.tex

\chapter{算法实现}
本文算法基于光滑自由变形\cite{Cui15},所以在变形、调整变形结果、绘制时所用的算法都与光滑自由变形相同。\autoref{fig:algorithm_ours}中是本文方法的流程图,可以与\autoref{fig:algorithm_sffd}对比发现,除了步骤3(切割三角形)采用了第\autoref{clip_algorithm}节中描述的算法,其它步骤均采用与光滑自由变形相同算法。

\begin{figure}
	\centering
    \tikzstyle{GPU} = [rectangle, draw, fill=blue!15, 
        text width=15em, text centered, rounded corners, minimum height=3em]
    
    \tikzstyle{CPU} = [rectangle, draw, fill=red!15, 
        text width=15em, text centered, rounded corners, minimum height=3em]
    \tikzstyle{line} = [draw, thick, ->, >= stealth]
    \begin{tikzpicture}[node distance = 2cm, auto]
        % Place nodes
        \node [CPU] (read) {1、输入多边形网格模型,并三角化};
        \node [CPU, below of=read] (initspace) {2、初始化B样条空间};
        \node [GPU, below of=initspace] (pntriangle) {3、求子三角面片的PN-Triangle,用以调整变形结果};
        \node [GPU, below of=pntriangle] (split) {4、用三角形均匀剖分算法分割三角形};
        \node [GPU, below of=split] (sample) {5、根据控制顶点,计算采样点的位置与法向};
        \node [GPU, below of=sample] (deformation) {6、用带约束的拟合的方法,计算出三角贝赛尔曲面片和法向量场的控制顶点};
        \node [GPU, below of=deformation] (adjust) {7、用法向信息和PN-Triangle信息调整上一步得到的控制顶点};
        \node [GPU, below of=adjust] (tess) {8、细分变形结果并绘制};
        \node [CPU, right=2em of deformation] (edit) {9、用户编辑控制顶点};

        % Draw edges
        \path [line] (read) -- (initspace);
        \path [line] (initspace) -- (pntriangle);
        \path [line] (pntriangle) -- (split);
        \path [line] (split) -- (sample);
        \path [line] (sample) -- (deformation);
        \path [line] (deformation) -- (adjust);
        \path [line] (adjust) -- (tess);
        \path [line] (tess) -| (edit);
        \path [line] (edit) |- (sample);
    \end{tikzpicture}
    \caption{本文算法流程图\\红色框表示在CPU中运行,蓝框表示在GPU中运行}\label{fig:algorithm_ours}
\end{figure}

\subsection{OpenGL Compute Shader实现}
CUDA是由英伟达推出的利用GPU进行通用计算的技术。该技术提供了方便易用的API,允许用户利用GPU强大的并行计算能力来加速算法,能得到了学术界和工业界的广泛应用。光滑自由变形\cite{Cui15}基于CUDA实现。大概获得了100多倍的加速。但是CUDA只能的在英伟达的硬件设备上运行,这大大限制了光滑自由变形的通用性,光滑自由变形不仅无法应用在其它的桌面平台的GPU(AMD、Intel核显)中,也无法在移动平台中运行。而后者由于计算资源有限,反而更加依赖GPU加速算法。

为了使我们的算法更加通用,我们将原先在光滑自由变形中GPU运算相关的代码从CUDA迁移到了OpenGL Compute Shader。由于OpenGL每次调用图形相关API与GPU通信时都会产生一定的CPU开销,并且将零散的GPU指令集中发送可以提供给GPU更大的优化空间。所以我们将步骤5至步骤8实现在了同一个Compute Shader中,不妨称之为Deform Shader。算法通过步骤9读取用户输入,再由Deform Shader对模型进行变形。这两个过程交替进行,以达到实时编辑模型的目的。

同样的原因,步骤3和步骤4也实现在同一个Compute Shader中,我们称之为Precompute Shader。其中步骤3中产生的结果将会用于步骤8中,以调整控制顶点位置。步骤4用本文算法对原始三角面片进行剖分,可以提高变形结果的精度。


\section{三角均匀剖分算法的GPU实现}
从\autoref{tab:clip_time_compare}与\autoref{tab:deformation_time_compare}中可以看出,光滑自由变形在按节点盒切割三角阶段所用时间非常久。虽然该过程只要在模型载入阶段执行一次,但是仍会对用户体验造成影响。光滑自由变形中,并末将这一过程用CUDA实现,主要是因为分割过程中需要随机的引用不同的数据,且控制流程较为复杂,不适合在GPU中实现。本文分割方式也具有以上特点,且还是一个递归的算法,所以同样不适合在GPU中实现。所以我们从另一个角度解决了这个问题,将本文提出的三角形均匀剖分算法实现在了GPU中,大大减少了算法的预计算时间。

首先,由于三角形均匀分割算法在计算三角形的某一条边需要被分成几段时,有一个取整操作。这就使得拥有不同但相近的边长的三角形,它们分割结果可能很相近,甚至一样。我们不妨将这些拥有相同分割方案的三角形称作“同类三角形”。同时,我们经过观察发现,对于同类三角形,可以共用一套分割方案,而不会引起子三角形质量的大幅下降。

基于以上观察,我们将第\autoref{clip_algorithm}章分成了两个阶段:计算分割方案、应用分割方案剖分三角形。


第一个阶段在CPU中进行,我们固定$l$为$1$,然后计算不同边长的三角形的分割方案,分割方案由两部分组成:子三角形的顶点在原三角形中的重心坐标、各个子三角形的顶点的连接关系。通过这些信息,可以方便的将一个三角形切割成多个子三角形。

计算分割方案需要较大的计算量以及复杂的分支判断,所以这部分计算在CPU中进行。又因为“同类三角形”的分割方案相同,所以这些分割方案可以提前预计算,以避免实际切割时重复计算。

实际分割过程几乎不会将一条边分成20段以上,因为过多的三角形无法显著提高变形结果的精度,反而会影响程序效率。所以在这一过程中,我们计算了所有三边长为${len_i}^{2}_{i=0}, 1 <= len_i <= 30, len_0 + len_1 > len_2, |len_0 - len_1| < len_2$的三角形的分割方案。并且分割方案以$\{len_i\}^{2}_{i=0}$为索引存储到一张查找表中。


第二个阶段在GPU中进行,先计算$\{\lceil len_i/l \rceil\}^{2}_{i=0}$,再以$\{\lceil len_i/l \rceil\}^{2}_{i=0}$为索引从查找表中找到分割方案,然后通过分割方案中分割点的重心坐标与原始三角面片的顶点位置为输入,可以用矩阵乘法直接求出分割点的位置。这些点的连接关系也由分割方案给出。上述过程没有递归,且分支较少,可以通过GPU并行计算架构获得较大的加速。

实际上,我们是原从三角均匀剖分算法中,将“同类三角形”重复计算分割方案的过程提了出来。并与GPU中实际分割过程解耦。然后运用以空间换取时间的方法加速了分割过程。不仅如此,解耦后算法的复杂度集中在了第一阶段,使得第二阶段可以用GPU加速。所以本文的切割算法比光滑自由变形中的算法快了近200倍,如\autoref{tab:clip_time_compare}所示。


但是在该算法中,“同类三角形”如果由原来的三角均匀剖分算法直接切割,所得到的子三角形并不一定相同。所以该方法会使子三角形的质量略微下降,但是相比其带来的优势,我们仍有理由采用该方法。

% !TEX root = ../main.tex
\chapter{算法在移动平台的实现}

本文方法用OpenGL Compute实现,所以可以在移动设备中实现。我们将该方法应用到了一款安卓软件\footnote{哇陶是一款陶瓷制作过程模拟软件,用户可以借助该软件参与到陶瓷制作的各个过程中(拉胚、烧制、贴花等)。}中,使得用户可以在拉胚阶段自由的编辑瓷器形状。


% !TEX root = ../main.tex
\chapter{结果对比}
本章主要进行了绘制效果,算法效率,算法精度这三方面的对比,我们进行对比实验的硬件条件如下:CPU为英特尔酷睿i7 4710MQ@2.50GHz,GPU为英伟达GeForce GT 730M,内存大小为8GB。对比的对象是传统的自由变形\cite{Sederberg86}和光滑自由变形\cite{Cui15}。选用的变形空间均为B样条体,其次数为$3\times3\times3$,控制顶点个数为$6\times6\times6$。为了方便比较,待变形的模型顶点都将先归一化到$[-1, 1]^3$

由于光滑自由变形和本文算法最终的变形结果均为三角Bézier曲面片,本文采取与光滑自由变形相同的方案可视化这些曲面片,即均匀细分成三角形然后通过OpenGL绘制。用户可以通过参数控制三角Bézier曲面片的细分粒度,以将解析结果细分成不同精度的三角网格模型。在本章的实验中,所有三角Bézier曲面片都将被均匀细分成100个小三角形进行绘制。同时在比较光滑自由变形和本文方法的拟合误差时,误差值也将通过比较细分后三角形顶点的属性得到的。

在以下对比实验中,本文方法在分割阶段时,$l$的值固定为$\frac{1}{3}$的节点盒长度。

\section{绘制效果比对}
本节主要对比传统自由变形、光滑自由变形以及本文算法的绘制效果。对比实验的硬件环境与算法参数如上文所述。实验中选用帆船作为对比模型,该模型中包含了各种大小、各种形状的三角形,能够有效的检验变形算法产生结果的质量。

\autoref{subfig:renderer_effect}中展示了各个方法的绘制结果,\autoref{subfig:renderer_effect1}是传统自由变形的结果,可以很明显的看到帆船桅杆部分因采样密度不足造成的走样问题。光滑自由变形(\autoref{subfig:renderer_effect3})和本文方法(\autoref{subfig:renderer_effect2})均很好的解决了这一问题。

同时通过比较\autoref{subfig:renderer_effect3}和\autoref{subfig:renderer_effect2},可以发现本文算法和光滑自由变形的结果,从视觉上基本无法区分。\autoref{subfig:renderer_effect4}是本文方法和光滑自由变形的像素层面上的差异。红色的深度越深表示差异越大。白色表示没有差异。\autoref{subfig:renderer_effect3}和\autoref{subfig:renderer_effect2}的像素用256阶灰度表示,除去背景像素,两者像素值平均差异是$0.249$,最大差异是$51$。其中灰度差异小于5的像素点了占了总像素点的$99.02\%$。所以肉眼几乎无法看出两者差别。

通过以上分析,可以认为本文方法达到了和光滑自由变形相同的绘制效果。
\begin{figure}[htbp]
	\centering
	\begin{subfigure}[b]{.4\textwidth}
		\centering
		\includegraphics[width = \textwidth]{renderer_effect0.png}
		\caption{原始模型}\label{subfig:renderer_effect0}
	\end{subfigure}
	\quad
	\begin{subfigure}[b]{.4\textwidth}
		\centering
		\includegraphics[width = \textwidth]{renderer_effect1.png}
		\caption{传统自由变形结果}\label{subfig:renderer_effect1}
	\end{subfigure}

	\centering
	\begin{subfigure}[b]{.4\textwidth}
		\centering
		\includegraphics[width = \textwidth]{renderer_effect3.png}
		\caption{光滑变形结果}\label{subfig:renderer_effect3}
	\end{subfigure}
	\quad
	\begin{subfigure}[b]{.4\textwidth}
		\centering
		\includegraphics[width = \textwidth]{renderer_effect2.png}
		\caption{本文方法结果}\label{subfig:renderer_effect2}
	\end{subfigure}

	\centering
	\begin{subfigure}[b]{.4\textwidth}
		\centering
		\includegraphics[width = \textwidth]{renderer_effect4.png}
        \caption{\autoref{subfig:renderer_effect3}和\autoref{subfig:renderer_effect2}的像素差异}\label{subfig:renderer_effect4}
	\end{subfigure}
	\quad
	\begin{subfigure}[b]{.4\textwidth}
		\centering
		\includegraphics[width = \textwidth]{renderer_effect5.png}
		\caption{带纹理的本文结果}\label{subfig:renderer_effect5}
	\end{subfigure}
	\caption{绘制效果对比图}\label{subfig:renderer_effect}
\end{figure}

\section{效率对比}
本节的比较的对象仍然是光滑自由变形,选用的模型是一个高精度的蜗牛模型,如\autoref{fig:speed_compare}所示,由46742个面片组成。

\begin{figure}[htbp]
	\centering
	\begin{subfigure}[b]{.4\textwidth}
		\centering
		\includegraphics[width = \textwidth]{snail1.png}
		\caption{光滑自由变形绘制结果}\label{subfig:snail1}
	\end{subfigure}
	\quad
	\begin{subfigure}[b]{.4\textwidth}
		\centering
		\includegraphics[width = \textwidth]{snail2.png}
		\caption{本文方法绘制结果}\label{subfig:snail2}
	\end{subfigure}
	\caption{效率对比采用的蜗牛模型}\label{fig:speed_compare}
\end{figure}

\subsection{分割阶段}
光滑自由变形的分割阶段在CPU中完成,本文方法在GPU中完成。\autoref{tab:clip_time_compare}中是两种方法在分割阶段所用的时间,本文方法具有较大优势,速度上快了近200倍。这其中在两方面的原因,一方面我们通过以空间换时间的策略,对较为耗时的“计算分割方案”这一过程进行了预计算。另一方面是我们用GPU实现了分割过程,而光滑自由变形是在CPU中实现的。

\begin{table}[htbp]
    \centering
    \caption{三角形分割阶段运行时间对比(单位:ms)}\label{tab:clip_time_compare}
    \begin{tabular}{lrr}
    \toprule
    \textbf{步骤}   & \textbf{本文方法} & \textbf{光滑自由变形\cite{Cui15}} \\
    \midrule
    生成PN三角形    & 4.618             & 104.465                           \\
    切割原始三角面片& 21.475            & 5027.418                          \\
    \midrule
    \textbf{总计}   & 26.093            & 5131.883                          \\
    \bottomrule
    \end{tabular}
\end{table}

\subsection{变形、绘制阶段}
变形过程又分为几个以下几个子过程:
\begin{enumerate}
    \item 计算拟合点。
    \item 计算约束点。
    \item 计算三角Bézier曲面片的控制顶点。
    \item 调整控制顶点。
\end{enumerate}

绘制也可以分为两个子过程:
\begin{enumerate}
    \item 离散化三角Bézier曲面片。
    \item 绘制。
\end{enumerate}

其中,绘制交由OpenGL流水管线实现,故不比较这一过程的时间。其它过程计算时间如\autoref{tab:deformation_time_compare}所示。

\begin{table}[htbp]
    \centering
    \caption{三角形变形、绘制阶段运行时间对比(单位:ms)}\label{tab:deformation_time_compare}
    \begin{tabular}{lrr}
    \toprule
    \textbf{步骤}     & \textbf{本文方法} & \textbf{光滑自由变形\cite{Cui15}} \\
    \midrule
    计算采样点位置     & \multirow{6}{*}{67.419} & 41.369     \\
    计算三角Bézier曲面片控制顶点      &                         & 11.125     \\
    计算三角Bézier曲面片的法向量场   &                         & 4.209     \\
    微调控制顶点         &                         & 4.344     \\
    离散三角Bézier曲面片 &                         & 4.057     \\
    \midrule
    \textbf{总计}& 67.419  & 65.464    \\
    \bottomrule
    \end{tabular}
\end{table}
本文采用OpenGL Compute Shader实现,为了尽可能提高运行效率,\autoref{tab:deformation_time_compare}中的几个阶段实现在了同一个Shader中,所以以上几个阶段本文方法只给出了总运行时间。

光滑自由变形由CUDA实现,该方法在实现上述过程时,大量的采用了大矩阵乘法,并交由cuBLAS计算。cuBLAS是CUDA官方提供的线性运算库,在相同的硬件下,实现相同的计数任务,用户自己编写的程序很难在时间效率上超过cuBLAS。所以我们有理由将光滑自由变形的变形过程所用的时间作为一个性能的标杆。本文以之为目标,作了很多算法及语言层面上的优化。最终得到的结果只比光滑自由变形慢了$2.99\%$。考虑到我们的算法采用的OpenGL Compute Shader实现,OpenGL Compute Shader并末提供大矩阵运算库,无法从大矩阵运算中得益,所以我们认为本文变形所用的时间是一个较为合理的结果。

\subsection{总结}
综上所述,在变形、绘制阶段,本文算法基本能达到和光滑自由变形相同的运行速度。而在分割阶段,本文方法具有较大优势,约比光滑自由变形快了近200倍。

\section{精度对比}
虽然光滑自由变形基于精确自由变形,但是该方法最终变形结果是通过用低次的三角Bézier曲面片拟合高次的精确结果得到,这样虽然提高了算法的运行效率,但同时也引入了法向和几何上的误差。本文方法也存在相同的拟合误差,并且本文未沿节点盒切割原始三角面片,这也是一个误差增加的潜在因素。本节希望通过误差分析说明本文方法的误差相较于自由变形并不会有显著增加。

本节中,本文方法的比较对象仍是光滑自由变形,我们分别计算光滑自由变形、本文方法的变形结果与精确结果的差异以得到两种方法对应的误差值。与光滑自由变形一样,本文方法也会通过法向信息微调三角Bézier曲面片的控制顶点。所以我们需要几何与法向都精确的模型作为本节实验变形所用的模型。为此我们选择了一个标准的立方体和一个由36个三次Bézier曲面组成的Utah茶壶作为变形模型进行误差分析,如\autoref{fig:error_ori}所示。标准立方体的精确变形结果可以通过精确自由变形\cite{Feng00}得到,Utah茶壶的精确变形结果可以通过传统FFD结合均匀加密采样得到。由于Utah茶壶是由Bézier曲面片组成的,所以必须先转化成多边形网格才能作为变形算法的输入,我们采用Casteljau细分算法\cite{farin2000essentials}将其转化成多边形网格。

\begin{figure}[htbp]
	\centering
	\begin{subfigure}[b]{.4\textwidth}
		\centering
		\includegraphics[width = \textwidth]{cube0.png}
		\caption{立方体}\label{subfig:cube0}
	\end{subfigure}
	\quad
	\begin{subfigure}[b]{.4\textwidth}
		\centering
		\includegraphics[width = \textwidth]{teapot0.png}
		\caption{Utah茶壶}\label{subfig:teapot0}
	\end{subfigure}
	\quad
	\caption{误差估计采用的模型}\label{fig:error_ori}
\end{figure}

\autoref{fig:cube_result}和\autoref{fig:teapot_result}是分别是立方体和Utah茶壶的在各个变形方法下的变形结果。

\begin{figure}[htbp]
	\centering
	\begin{subfigure}[b]{.3\textwidth}
		\centering
		\includegraphics[width = \textwidth]{cube1.png}
		\caption{精确自由变形结果}\label{subfig:cube1}
	\end{subfigure}
	\begin{subfigure}[b]{.3\textwidth}
		\centering
		\includegraphics[width = \textwidth]{cube2.png}
		\caption{光滑自由变形结果}\label{subfig:cube2}
	\end{subfigure}
	\begin{subfigure}[b]{.3\textwidth}
		\centering
		\includegraphics[width = \textwidth]{cube3.png}
		\caption{本文结果}\label{subfig:cube3}
	\end{subfigure}
	\caption{立方体变形结果}\label{fig:cube_result}
\end{figure}

\begin{figure}[htbp]
	\centering
	\begin{subfigure}[b]{.3\textwidth}
		\centering
		\includegraphics[width = \textwidth]{teapot1.png}
		\caption{传统自由变形结果}\label{subfig:teapot1}
	\end{subfigure}
	\begin{subfigure}[b]{.3\textwidth}
		\centering
		\includegraphics[width = \textwidth]{teapot2.png}
		\caption{光滑自由变形结果}\label{subfig:teapot2}
	\end{subfigure}
	\begin{subfigure}[b]{.3\textwidth}
		\centering
		\includegraphics[width = \textwidth]{teapot3.png}
		\caption{本文结果}\label{subfig:teapot3}
	\end{subfigure}
	\caption{Utah茶壶变形结果}\label{fig:teapot_result}
\end{figure}

\autoref{subfig:cube1}是立方体模型通过精确自由变形得到的结果,该结果是精确的,所以光滑自由变形和本文方法得到的结果(如图\autoref{subfig:cube2}与\autoref{subfig:cube3}所示)都与精确自由变形得到的结果进行比较,以得到两者的变形误差,结果如\autoref{tab:error_cube}所示。其中,几何误差定义为对应点的欧氏距离,法向误差定义为对应点法向所成夹角的角度。从\autoref{tab:error_cube}中可以看出,本文方法的几何误差与法向误差均小于光滑自由变形,主要原因是本文方法选取的$l$比较小,产生的子三角形要比光滑自由变形小很多。\autoref{fig:cube_error0}、\autoref{fig:cube_error1}中是上述误差的可视化结果。

\begin{table}[htbp]
    \centering
    \caption{立方体误差} \label{tab:error_cube}
    \begin{tabular}{lrrrr}
    \toprule
                    & 平均几何误差 & 最大几何误差 & 平均法向误差 & 最大法向误差 \\
    \midrule
        本文方法    & \num{0.002952310} & \num{0.02908119} & \num[scientific-notation=false]{0.5785680}$^\circ$ & \num[scientific-notation=false]{15.89719}$^\circ$ \\
        光滑自由变形& \num{0.004904391} & \num{0.03718415} & \num[scientific-notation=false]{0.5853341}$^\circ$ & \num[scientific-notation=false]{21.24638}$^\circ$ \\
    \bottomrule
    \end{tabular}
\end{table}

\begin{figure}[htbp]
	\centering
	\begin{subfigure}[b]{.45\textwidth}
		\centering
		\includegraphics[width = \textwidth]{cube4.png}
		\caption{本文方法几何误差图}\label{subfig:cube4}
	\end{subfigure}%
	\begin{subfigure}[b]{.45\textwidth}
		\centering
		\includegraphics[width = \textwidth]{cube6.png}
		\caption{光滑自由变形几何误差图}\label{subfig:cube6}
	\end{subfigure}
	\caption{立方体几何误差}\label{fig:cube_error0}
\end{figure}
\begin{figure}[htbp]
	\centering
	\begin{subfigure}[b]{.45\textwidth}
		\centering
		\includegraphics[width = \textwidth]{cube5.png}
		\caption{本文方法法向误差图}\label{subfig:cube5}
	\end{subfigure}%
	\begin{subfigure}[b]{.45\textwidth}
		\centering
		\includegraphics[width = \textwidth]{cube7.png}
		\caption{光滑自由变形法向误差图}\label{subfig:cube7}
	\end{subfigure}
	\caption{立方体法向误差}\label{fig:cube_error1}
\end{figure}

对于Utah茶壶,对比结果如\autoref{tab:error_utah}所示。我们同样给出了上述误差的可视化结果,如\autoref{fig:teapot_error0}、\autoref{fig:teapot_error1}所示。Utah茶壶经过Casteljau细分算法得到的三角形本身比较小,因此本文方法中较小的$l$没有优势,所以两种结果的几何误差和法向误差较为接近。同时从\autoref{subfig:teapot5}和\autoref{subfig:teapot7}可以看出,无论上本文方法还是光滑自由变形,曲率较大的地方法向误差都会比较大。

\begin{table}[htbp]
    \centering
    \caption{Utah茶壶误差} \label{tab:error_utah}
    \begin{tabular}{lrrrr}
    \toprule
        & 平均几何误差 & 最大几何误差 & 平均法向误差 & 最大法向误差 \\
    \midrule
        本文方法    & \num{0.006597950} & \num{0.01203453} & \num[scientific-notation=false]{0.6152188}$^\circ$ & \num[scientific-notation=false]{22.53011}$^\circ$ \\
        光滑自由变形& \num{0.006650957} & \num{0.01220984} & \num[scientific-notation=false]{0.5491496}$^\circ$ & \num[scientific-notation=false]{22.53010}$^\circ$ \\
    \bottomrule
    \end{tabular}
\end{table}

\begin{figure}[htbp]
	\centering
	\begin{subfigure}[b]{.45\textwidth}
		\centering
		\includegraphics[width = \textwidth]{teapot4.png}
		\caption{本文方法几何误差图}\label{subfig:teapot4}
	\end{subfigure}%
	\begin{subfigure}[b]{.45\textwidth}
		\centering
		\includegraphics[width = \textwidth]{teapot6.png}
		\caption{光滑自由变形几何误差图}\label{subfig:teapot6}
	\end{subfigure}
	\caption{Utah茶壶几何误差}\label{fig:teapot_error0}
\end{figure}
\begin{figure}[htbp]
	\centering
	\begin{subfigure}[b]{.45\textwidth}
		\centering
		\includegraphics[width = \textwidth]{teapot5.png}
		\caption{本文方法法向误差图}\label{subfig:teapot5}
	\end{subfigure}%
	\begin{subfigure}[b]{.45\textwidth}
		\centering
		\includegraphics[width = \textwidth]{teapot7.png}
		\caption{光滑自由变形法向误差图}\label{subfig:teapot7}
	\end{subfigure}
	\caption{Utah茶壶法向误差}\label{fig:teapot_error1}
\end{figure}

\section{本章小结}
本章从绘制效果、时间效率、变形误差这三个方面对比了本文算法与同类算法。从对比结果中我们可以发现,本文算法达到了和光滑自由变形相同的变形效果,并在速度上相对于光滑自由变形算法具有一定的优势。本文算法各方面的性能均达到了我们的预期。

\chapter{总结与展望}
\section{本文工作总结}
    空间变形作为计算机图形学领域的热点,经过几十年的发展,可以说已经相当成熟了。前人的工作在这一领域大量的研究工作,无论是在理论深度还是应用广度,都取得了丰硕的研究成果,足以应对绝大部分需求。但是用户的需求也在不断的发生变化。

    在之前,可能只有游戏开发,动画特效等专业领域对三维模型编辑有较大的需求,变形算法的用户大都比较有经验,其硬件设备也相对高端。相应的,变形算法可能更多的是考虑变形效果,变形算法的表现力等;而不怎么关注变形算法的效率及交互复杂度。

    而随着硬件基础的提升,普通用户对内容的丰富程度有了更高的要求。这就如同互联网在其发展初期,内容的主要载体是文字。但是发展到现阶段,互联网内容的载体早已不再局限于单纯的文字,图片,音频,视频等多媒体的资源正占据着越来越高的比重。在可以预见的将来,三维内容也将成为一种主要的内容载体。同时,用户在互联网中的身份也慢慢从单纯的消费者转变为兼具消费和生产双重属性的角色。再这样的趋势下,普通用户也会产生大量的三维模型编辑需求。而这部分用户,可能更多的关注变形算法的效率\footnote{可能的硬件及其有限的环境中运行,如手机,平板电脑。}、以及算法的易用性\footnote{用户可能不怎么关心模型质量,而对交互有较高的要求。}

    所以,这一成熟的领域仍需要有新的工作满足不断出现的新需求。

    我们的工作就是在这样的环境下提出的,相对于一般的自由变形,具有如下几点优势:
    \begin{itemize}
        \item 变形时考虑法向信息,变形结果光滑自然。
        \item 算法用OpenGL实现,利益于OpenGL跨平台的特性,本文算法可以运行在各个平台下。
        \item 算法通过OpenGL Compute Shader实现变形过程中的几乎所有步骤\footnote{读取模型与初始化变形空间除外,前者只能由CPU完成,后者计算量很小,没必要用GPU加速。},使得算法在主流硬件设备中都能实时交互。
        \item 用户可以通过改变切割三角形的大小调整算法的效率和变形质量,以满足不同用户的需求。
    \end{itemize}

    本文算法基于崔的光滑自由变形\cite{Cui15}实现。算法继承了光滑自由变形的主要优势:1)变形效果光滑自然;2)算法效率高,可实时交互。但是针对光滑自由变形中的不足之处,我们主要作了如下两个方面的改进:
    \begin{itemize}
        \item 改进三角形部分算法,使该过程更加鲁棒、高效。
        \item 改用OpenGL实现GPU并行计算,本文算法在保持保有的运行效率的前提下大大提高算法的通用性。
    \end{itemize}

    文章最后的实验结果表明了本文算法很好的满足了以上预期。不但如此,我们还将算法应用到了生产环境中并取得了良好的效果。

\section{未来工作展望}
    本文工作还有很多有待改进之处。

    其中较为突出的一点是算法的造型能力不足,如用户基本不可能用我们的算法产生如毛发、鳞片之间的细节。尽管基于体的空间变形拥有较高的自由度,但是很多控制顶点往往在模型之内,作为用户很难有效的利用这些控制顶点带来的自由度。增加控制顶点能在一定程度上提升造型能力,但是这会使算法交互变的复杂。在未来的工作中我们可以尝试不同的变形空间,以便用户实现更加细致的变形意图。

    其次,本文的切割算法需要一定的空间,这一空间的大小和预计算的分割方案的数目呈平方关系,会随预计算的存储分割方案的增加急剧增加\footnote{本文工作中我们预计算了边长分割成30段及以下的所有三角形的分割方案,需要约13.5MB的内存空间。如果我们预计算了边长分割成50段及以下的所有三角形的分割方案,所需内存空间将增长到137.9MB}。所以当前算法无法处理分割段数较大的三角形。否则,内存会有较大开销。在未来的工作中,我们可以考虑进一步优化分割方案的存储,以减小预计算所产生的开销。




\backmatter{}
\bibliography{reference_data_base/references}
%\include{contents/thanks}
%% !TEX root = ../main.tex
\chapter{作者介绍}
陆哲琪,男,生于1991年11月,浙江舟山人。2014年毕业于重庆大学计算机科学与技术学院,获得学士学位。2014年9月至今在浙江大学攻读硕士学位。现在浙江大学计算机辅助与图形学国家重点实验室。研究方向为计算机图形学,对计算机数学几何建模、自由变形等兴趣浓厚。

% \nocite{*} % to show the entire references, annotate it if need.
%\appendix
%\include{contents/appendixA}
%\include{contents/appendixB}
\end{document}
