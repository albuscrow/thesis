% !TEX root = ../main.tex

\chapter{基于OpenGL Compute Shader实现算法}
CUDA是由英伟达推出的利用GPU进行通用计算的技术。该技术提供了方便易用的API,允许用户利用GPU强大的并行计算能力来加速算法,能得到了学术界和工业界的广泛应用。光滑自由变形\cite{Cui15}基于CUDA实现。大概获得了100多倍的加速。但是CUDA只能的在英伟达的硬件设备上运行,这大大限制了光滑自由变形的通用性,光滑自由变形不仅无法应用在其它的桌面平台的GPU(AMD、Intel核显)中,也无法在移动平台中运行。而后者由于计算资源有限,反而对GPU加速算法更加依赖。

所以,我们采用了OpenGL的Comput Shader实现本文算法,使得算法更加通用。在具体实现过程中,变形、微调三角贝赛尔曲面片(变形结果)的控制顶点,细分三角贝赛尔曲面片均采用和光滑自由变形相同的算法。不仅如此,我们还将光滑自由变形中CPU实现了部分——分割过程,实现在了GPU中,使整个算法都在GPU中完成。

\section{三角均匀剖分算法的GPU实现}
第\autoref{clip_algorithm}节中的算法是一个递归调用的算法,且算法中还有很多判断分支和特殊情况。这种算法不适合用GPU实现。所以,我们采用以空间换取时间并控制复杂度的策略,先预计算了不同边长的三角形的分割方案,并将分割方案以三角面片各边被分割的段数,即$\{\lceil len_i/l \rceil\}^{2}_{i=0}$为索引储存在一张表中,分割方案由分割点的重心坐标与子三角形的连接情况表示。当Compute Shader需要分割三角面片时,先计算$\{\lceil len_i/l \rceil\}^{2}_{i=0}$,再以此为索引找出合适的分割方案,进而完成三角形的分割。

在Compute Shader分割三角形时,以分割点的重心坐标与三角面片的顶点位置为输入,可以用矩阵乘法直接求出分割点的位置。这些点的连接关系也由分割方案给出。上述过程没有递归,且分支较少,很适合GPU实现。

但是在该算法中,不同的三角形,其索引值$\{\lceil len_i/l \rceil\}^{2}_{i=0}$可能相同。也就是说不同的三角形,只要其索引值$\{\lceil len_i/l \rceil\}^{2}_{i=0}$相同,那么它们的分割方案都相同,由连长为$\{\lceil len_i/l \rceil * l\}^{2}_{i=0}$的三角面片预计算确定。虽然这会在一定程度上带来子三角形质量的下降,但考虑到其对算法性能的提升,这一方法仍是值得采用的。\textcolor{red}{此处需要具体数据}


\section{在移动设备上的实现}
本文方法用OpenGL Compute实现,所以可以在移动设备中实现。我们将该方法应用到了一款安卓软件\footnote{哇陶是一款陶瓷制作过程模拟软件,用户可以借助该软件参与到陶瓷制作的各个过程中(拉胚、烧制、贴花等)。}中,使得用户可以在拉胚阶段自由的编辑瓷器形状。

