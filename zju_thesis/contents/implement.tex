% !TEX root = ../main.tex

\chapter{算法实现}
本文算法基于光滑自由变形\cite{Cui15},所以在变形、调整变形结果、绘制时所用的算法都与光滑自由变形相同。\autoref{fig:algorithm_ours}中是本文方法的流程图,可以与\autoref{fig:algorithm_sffd}对比发现,除了步骤3(切割三角形)采用了第\autoref{clip_algorithm}节中描述的算法,其它步骤均采用与光滑自由变形相同算法。

\begin{figure}
	\centering
    \tikzstyle{GPU} = [rectangle, draw, fill=blue!15, 
        text width=15em, text centered, rounded corners, minimum height=3em]
    
    \tikzstyle{CPU} = [rectangle, draw, fill=red!15, 
        text width=15em, text centered, rounded corners, minimum height=3em]
    \tikzstyle{line} = [draw, thick, ->, >= stealth]
    \begin{tikzpicture}[node distance = 2cm, auto]
        % Place nodes
        \node [CPU] (read) {1、输入多边形网格模型,并三角化};
        \node [CPU, below of=read] (initspace) {2、初始化B样条空间};
        \node [GPU, below of=initspace] (pntriangle) {3、求子三角面片的PN-Triangle,用以调整变形结果};
        \node [GPU, below of=pntriangle] (split) {4、用三角形均匀剖分算法分割三角形};
        \node [GPU, below of=split] (sample) {5、根据控制顶点,计算采样点的位置与法向};
        \node [GPU, below of=sample] (deformation) {6、用带约束的拟合的方法,计算出三角贝赛尔曲面片和法向量场的控制顶点};
        \node [GPU, below of=deformation] (adjust) {7、用法向信息和PN-Triangle信息调整上一步得到的控制顶点};
        \node [GPU, below of=adjust] (tess) {8、细分变形结果并绘制};
        \node [CPU, right=2em of deformation] (edit) {9、用户编辑控制顶点};

        % Draw edges
        \path [line] (read) -- (initspace);
        \path [line] (initspace) -- (pntriangle);
        \path [line] (pntriangle) -- (split);
        \path [line] (split) -- (sample);
        \path [line] (sample) -- (deformation);
        \path [line] (deformation) -- (adjust);
        \path [line] (adjust) -- (tess);
        \path [line] (tess) -| (edit);
        \path [line] (edit) |- (sample);
    \end{tikzpicture}
    \caption{本文算法流程图\\红色框表示在CPU中运行,蓝框表示在GPU中运行}\label{fig:algorithm_ours}
\end{figure}

\subsection{OpenGL Compute Shader实现}
CUDA是由英伟达推出的利用GPU进行通用计算的技术。该技术提供了方便易用的API,允许用户利用GPU强大的并行计算能力来加速算法,能得到了学术界和工业界的广泛应用。光滑自由变形\cite{Cui15}基于CUDA实现。大概获得了100多倍的加速。但是CUDA只能的在英伟达的硬件设备上运行,这大大限制了光滑自由变形的通用性,光滑自由变形不仅无法应用在其它的桌面平台的GPU(AMD、Intel核显)中,也无法在移动平台中运行。而后者由于计算资源有限,反而更加依赖GPU加速算法。

为了使我们的算法更加通用,我们将原先在光滑自由变形中GPU运算相关的代码从CUDA迁移到了OpenGL Compute Shader。由于OpenGL每次调用图形相关API与GPU通信时都会产生一定的CPU开销,并且将零散的GPU指令集中发送可以提供给GPU更大的优化空间。所以我们将步骤5至步骤8实现在了同一个Compute Shader中,不妨称之为Deform Shader。算法通过步骤9读取用户输入,再由Deform Shader对模型进行变形。这两个过程交替进行,以达到实时编辑模型的目的。

同样的原因,步骤3和步骤4也实现在同一个Compute Shader中,我们称之为Precompute Shader。其中步骤3中产生的结果将会用于步骤8中,以调整控制顶点位置。步骤4用我们的三角形均匀剖分算法,对原始三角面片进行剖分,可以提高变形结果的精度。


\section{三角均匀剖分算法的GPU实现}
从\autoref{tab:clip_time_compare}与\autoref{tab:deformation_time_compare}中可以看出,光滑自由变形在按节点盒切割三角阶段所用时间非常久。虽然该过程只要在模型载入阶段执行一次,但是仍会对用户体验造成影响。光滑自由变形中,并末将这一过程用CUDA实现,主要是因为分割过程中需要随机的引用不同的数据,且控制流程较为复杂,不适合在GPU中实现。本文分割方式也具有以上特点,且还是一个递归的算法,所以同样不适合在GPU中实现。所以我们从另一个角度解决了这个问题,将本文提出的三角形均匀剖分算法实现在了GPU中,大大减少了算法的预计算时间。

首先,由于三角形均匀分割算法在计算三角形的某一条边需要被分成几段时,有一个取整操作。这就使得拥有不同但相近的边长的三角形,它们分割结果可能很相近,甚至一样。我们不妨将这些拥有相同分割方案的三角形称作“同类三角形”。同时,我们经过观察发现,对于同类三角形,可以共用一套分割方案,而不会引起子三角形质量的大幅下降。

基于以上观察,我们将第\autoref{clip_algorithm}章分成了两个阶段:计算分割方案、应用分割方案剖分三角形。


第一个阶段在CPU中进行,我们固定$l$为$1$,然后计算不同边长的三角形的分割方案,分割方案由两部分组成:子三角形的顶点在原三角形中的重心坐标、各个子三角形的顶点的连接关系。通过这些信息,可以方便的将一个三角形切割成多个子三角形。

计算分割方案需要较大的计算量以及复杂的分支判断,所以这部分计算在CPU中进行。又因为“同类三角形”的分割方案相同,所以这些分割方案可以提前预计算,以避免实际切割时重复计算。

实际分割过程几乎不会将一条边分成20段以上,因为过多的三角形无法显著提高变形结果的精度,反而会影响程序效率。所以在这一过程中,我们计算了所有三边长为${len_i}^{2}_{i=0}, 1 <= len_i <= 30, len_0 + len_1 > len_2, |len_0 - len_1| < len_2$的三角形的分割方案。并且分割方案以$\{len_i\}^{2}_{i=0}$为索引存储到一张查找表中。


第二个阶段在GPU中进行,先计算$\{\lceil len_i/l \rceil\}^{2}_{i=0}$,再以$\{\lceil len_i/l \rceil\}^{2}_{i=0}$为索引从查找表中找到分割方案,然后通过分割方案中分割点的重心坐标与原始三角面片的顶点位置为输入,可以用矩阵乘法直接求出分割点的位置。这些点的连接关系也由分割方案给出。上述过程没有递归,且分支较少,可以通过GPU并行计算架构获得较大的加速。

实际上,我们是原从三角均匀剖分算法中,将“同类三角形”重复计算分割方案的过程提了出来。并与GPU中实际分割过程解耦。然后运用以空间换取时间的方法加速了分割过程。不仅如此,解耦后算法的复杂度集中在了第一阶段,使得第二阶段可以用GPU加速。所以本文的切割算法比光滑自由变形中的算法快了近200倍,如\autoref{tab:clip_time_compare}所示。


但是在该算法中,“同类三角形”如果由原来的三角均匀剖分算法直接切割,所得到的子三角形并不一定相同。所以该方法会使子三角形的质量略微下降,但是相比其带来的优势,我们仍有理由采用该方法。
