\chapter{总结与展望}
\section{本文工作总结}
    空间变形作为计算机图形学领域的热点,经过几十年的发展,可以说已经相当成熟了。前人的工作在这一领域大量的研究工作,无论是在理论深度还是应用广度,都取得了丰硕的研究成果,足以应对绝大部分需求。但是用户的需求也在不断的发生变化。

    在之前,可能只有游戏开发,动画特效等专业领域对三维模型编辑有较大的需求,变形算法的用户大都比较有经验,其硬件设备也相对高端。相应的,变形算法可能更多的是考虑变形效果,变形算法的表现力等;而不怎么关注变形算法的效率及交互复杂度。

    而随着硬件基础的提升,普通用户对内容的丰富程度有了更高的要求。这就如同互联网在其发展初期,内容的主要载体是文字。但是发展到现阶段,互联网内容的载体早已不再局限于单纯的文字,图片,音频,视频等多媒体的资源正占据着越来越高的比重。在可以预见的将来,三维内容也将成为一种主要的内容载体。同时,用户在互联网中的身份也慢慢从单纯的消费者转变为兼具消费和生产双重属性的角色。再这样的趋势下,普通用户也会产生大量的三维模型编辑需求。而这部分用户,可能更多的关注变形算法的效率\footnote{可能的硬件及其有限的环境中运行,如手机,平板电脑。}、以及算法的易用性\footnote{用户可能不怎么关心模型质量,而对交互有较高的要求。}

    所以,这一成熟的领域仍需要有新的工作满足不断出现的新需求。

    我们的工作就是在这样的环境下提出的,相对于一般的自由变形,具有如下几点优势:
    \begin{itemize}
        \item 变形时考虑法向信息,变形结果光滑自然。
        \item 算法用OpenGL实现,利益于OpenGL跨平台的特性,本文算法可以运行在各个平台下。
        \item 算法通过OpenGL Compute Shader实现变形过程中的几乎所有步骤\footnote{读取模型与初始化变形空间除外,前者只能由CPU完成,后者计算量很小,没必要用GPU加速。},使得算法在主流硬件设备中都能实时交互。
        \item 用户可以通过改变切割三角形的大小调整算法的效率和变形质量,以满足不同用户的需求。
    \end{itemize}

    本文算法基于崔的光滑自由变形\cite{Cui15}实现。算法继承了光滑自由变形的主要优势:1)变形效果光滑自然;2)算法效率高,可实时交互。但是针对光滑自由变形中的不足之处,我们主要作了如下两个方面的改进:
    \begin{itemize}
        \item 改进三角形部分算法,使该过程更加鲁棒、高效。
        \item 改用OpenGL实现GPU并行计算,本文算法在保持保有的运行效率的前提下大大提高算法的通用性。
    \end{itemize}

    文章最后的实验结果表明了本文算法很好的满足了以上预期。不但如此,我们还将算法应用到了生产环境中并取得了良好的效果。

\section{未来工作展望}
    本文工作还有很多有待改进之处。

    其中较为突出的一点是算法的造型能力不足,如用户基本不可能用我们的算法产生如毛发、鳞片之间的细节。尽管基于体的空间变形拥有较高的自由度,但是很多控制顶点往往在模型之内,作为用户很难有效的利用这些控制顶点带来的自由度。增加控制顶点能在一定程度上提升造型能力,但是这会使算法交互变的复杂。在未来的工作中我们可以尝试不同的变形空间,以便用户实现更加细致的变形意图。

    其次,本文的切割算法需要一定的空间,这一空间的大小和预计算的分割方案的数目呈平方关系,会随预计算的存储分割方案的增加急剧增加\footnote{本文工作中我们预计算了边长分割成30段及以下的所有三角形的分割方案,需要约13.5MB的内存空间。如果我们预计算了边长分割成50段及以下的所有三角形的分割方案,所需内存空间将增长到137.9MB}。所以当前算法无法处理分割段数较大的三角形。否则,内存会有较大开销。在未来的工作中,我们可以考虑进一步优化分割方案的存储,以减小预计算所产生的开销。


