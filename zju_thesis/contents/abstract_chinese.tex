% !TEX root = ../main.tex

% 定义中文摘要和关键字
\begin{cabstract}

    在几何建模和计算机动画中,空间变形是几何外形编辑和柔性体动画生成的关键技术之一。其中,最具代表性的技术是自由变形。由于其简单易用、功能强大,已经被集成到很多商业软件中,例如:3DS Max、 Maya、 Softimage XSI等。本文针对目前自由变形方法中存在的几何采样、算法的通用性和鲁棒性等问题,提出了改进的光滑自由变形方法。

    首先,在光滑自由变形的框架下,提出了一种新的三角形分割算法----均匀剖分算法。该算法在模型空间而不是样条的参数空间中对三角形进行剖分,使产生的子三角形的形状尽可能接近正三角形,且它们的面积尽可能相等。与已有的方法相比,新方法可以在GPU中并行实现,并且使得变形算法鲁棒。

    其次,针对传统加速算法依赖于NVIDIA GPU的局限性,本文提出了基于OpenGL的改进光滑自由变形方法,采用OpenGL中通用计算功能Compute Shader实现算法中的计算密集部分。在保证算法通用性的同时,尽可能地提高了算法的计算效率。同时,我们将本文算法应用于移动平台的三维建模应用----“哇陶”中,验证了方法的跨平台通用性,探索了在资源有限的硬件环境中,自由变形算法实现三维模型外形编辑的可行性。

    最后,我们将本文算法与精确自由变形、光滑自由变形等方法进行了全方位的对比和分析,证明了本文方法可以在生成高质量物体变形的前提下,具有高效、鲁棒和平台独立的优势,扩展了自由变形方法的应用范围。

\end{cabstract}

\ckeywords{光滑自由变形,保留尖锐特征,OpenGL Compute Shader,GPU 加速,法向量场,三角形均匀剖分}
