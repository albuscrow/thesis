% !TEX root = ../main.tex

% 定义中文摘要和关键字
\begin{cabstract}
    介绍自由变形
    在崔的光滑自由变形中\cite{Cui15},算法在预处理阶段,会沿变形空间的结点盒对待变形模型形进行裁剪。但是裁剪可能会产生狭长多边形甚至退化多边形,这些多边形不仅会浪费计算资源,还会带来精度和程序鲁棒性的问题。另一方面,光滑自由变形使用了CUDA做并行计算,以保证变形算法的实时性,但是CUDA只能在Nvidia的平台止运行,这大限制了光滑自由变形的通用性。所以本文提出了一种基于光滑自由变形的新的变形算法,针对上述两个问题做出了相应的改进。首先,本文提出了一种新的多边形裁剪算法,使得裁剪产生的三角形连长尽可能相等。其次,本文用OpenGL的Compute Shader重新实现了算法框架,使之可以运行的不同的平台上。最后,本文还的Android平台实现了一个应用该算法的Application,以验证算法在计算资源较匮乏的情境下的表现。
\end{cabstract}

\ckeywords{光滑自由变形,保留尖锐特征,OpenGL 计算着色器,图形处理单元,法向量场}
