% !TEX root = ../main.tex

% 定义中文摘要和关键字
\begin{cabstract}

    在几何建模和计算机动画中,空间变形是几何外形编辑和柔性体动画生成的关键技术之一。其中,最具代表性的是自由变形技术,由于其简单易用、功能强大,已经被集成到很多商业软件中,例如:3DS Max、 Maya、 Softimage XSI等。本文针对目前自由变形方法中存在的几何采样、算法的通用性和鲁棒性问题,提出了改进的光滑自由变形方法。

    首先,本文提出了一种新的三角形分割算法----三角均匀剖分算法。该算法以任意三角形为输入,对三角形进行分割,使产生的子三角形的形状尽可能接近正三角形且它们的面积尽可能相等。此外,本文还对新旧两种分割算法进行了比较,并探索了新的分割算法对变形质量,运行效率的影响。

    其次,针对传统加速算法依赖于NVIDIA GPU的局限性,本文提出了基于OpenGL的改进光滑自由变形方法,采用OpenGL中通用计算功能Compute Shader实现算法中的计算密集部分。在保证算法通用性的同时,尽可能地提高了算法的计算效率。同时,我们将本文算法应用于移动平台的三维建模应用――“哇陶”中,验证了方法的跨平台通用性,探索了在资源有限的硬件环境中,自由变形算法实现三维模型外形编辑的可行性。

    最后,我们将本文算法与精确自由变形、光滑自由变形等方法进行了多方面的对比,以体现本文方法的优势。


\end{cabstract}

\ckeywords{光滑自由变形,保留尖锐特征,OpenGL Compute Shader,GPU 加速,法向量场,三角形均匀剖分}
