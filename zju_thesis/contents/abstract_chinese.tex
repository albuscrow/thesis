% !TEX root = ../main.tex

% 定义中文摘要和关键字
\begin{cabstract}
    自由变形\cite{Sederberg86}(Free Form Deformation,简称FFD)是一类编辑几何模型、生成柔体性动画的方法。由于其实现简单、与模型表示无关,所以被广泛的应用于计算机辅助设计与工程、计算机动画、医学影像等领域。另一方面,该工作不仅实现了一种具体的编辑三维模型的方法,更重要的是提出了一套解决三维模型编辑问题的框架,因此FFD\cite{Sederberg86}自提出以来就受到了学术届与工业界的广泛关注。众多研究人员在该框架基础上,提出了大量方法以改进原有的FFD的一些不足之处,如变形结果走样,交互繁琐,效率低下等。

    Cui等人的光滑自由变形\cite{Cui15}也是基于FFD框架的一种空间变形算法,其主要在算法效率、变形质量这两方面对前人的工作作出了改进。使得用户可以实时的编辑三维模型,且能得到光滑细腻的变形结果。但该方法仍存在一些问题:1)算法在预处理阶段,会沿变形空间的节点盒对待变形模型进行裁剪。但是裁剪可能会产生狭长三角形甚至退化三角形,这些三角形不仅会浪费计算资源,还会带来精度和程序鲁棒性等问题。2)光滑自由变形使用CUDA进行并行计算,以提升算法效率,保证变形算法可实时交互,但是CUDA只能在NVIDIA的平台上运行,这大大限制了光滑自由变形的通用性。

    因此,本文提出了一种基于光滑自由变形的新算法,针对上述问题做出了相应的改进。

    首先,本文提出了一种新的三角形分割算法----三角均匀剖分算法。该算法以任意三角形为输入,对三角形进行分割,使产生的子三角形的形状尽可能接近正三角形且它们的面积尽可能相等。此外,本文还对新旧两种分割算法进行了比较,并探索了新的分割算法对变形质量,运行效率的影响。

    其次,本文用OpenGL的Compute Shader实现了本文算法中除初始化之外的所有部分,在保证算法通用性的同时,尽可能地提高了算法的计算效率。同时,为了验证本文算法在不同平台下的通用性以及算法在计算资源较匮乏的情境下的表现,我们还在一个安卓平台的应用中应用了本文算法,并取得了较好的效果。

    最后,我们将本文算法与精确自由变形、光滑自由变形等方法进行了多方面的对比,以体现本文方法的优势。
\end{cabstract}

\ckeywords{光滑自由变形,保留尖锐特征,OpenGL Compute Shader,GPU 加速,法向量场,三角形均匀剖分}
