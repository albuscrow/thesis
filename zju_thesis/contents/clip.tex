% !TEX root = ../main.tex

\chapter{三角均匀剖分算法}
如上文所述,光滑自由变形需将模型沿节点盒切割,并将非三角形的面片三角化。这一过程很可能产生狭长三角形或者蜕化三角形,如\autoref{subfig:clip_compare0}所示,颜色越红表示三角形质量越差。过多的此类三角形不仅会浪费计算资源,还可能带来其它数值计算方面的问题。

另一方面,沿节点盒切割只是为了保证变形结果是严格的$n$次三角贝赛尔曲面片。若略去这一步骤,算法仍能继续,只不过结果会变得不精确。考虑到在光滑自由变形中,作者采用以精度换取效率的策略,用三次的三角贝赛尔曲面来拟合n次的精确结果,也就是说光滑自由变形的结果只是精确结果的近似。因此,光滑自由变形中“沿节点盒切割以保证结果精确”这一步骤就显的可有可无了。

而且,我们通过更进一步的观察发现,在光滑自由变形中,跨多个节点盒的三角面片只要足够小,其变形后在精度上的误差,相对于在节点盒内的三角面片而言,并不会显著增加。

因此,本文尝试提出一种更好的三角形分割算法,以替换光滑自由变形中的“沿节点盒切割”这一步骤。新算法不再沿节点盒切割三角面片,而是将三角面片按以下两个要求分割成子三角形:
\begin{itemize}
    \item 所有子三角形的三边长都尽可能相等。以得到形状尽可能接近正三角形,且面积尽可能相同的子三角形。
    \item 所有子三角形的顶点均不可位于其它子三角形的边上。以避免变形后产生裂缝。
\end{itemize}

我们的新算法相比原来的沿节点盒分割的算法而言有两点优势:
\begin{itemize}
        \item 切分出来的子三角形尽可能“正”,不会产生新的狭长或蜕化的三角形。
        \item 参数$l$使得用户可以控制产生的子三角形的大小。
\end{itemize}


\section{算法实现}\label{clip_algorithm}
首先我们先定义一些符号以方便描述算法实现。$l$,$t$为算法的输入,$l$表示子三角形边长的期望值,算法分割产生的子三角形的三边的长度需尽可能接近$l$。$t$表示待分割的三角面片。$\{e_i\}^{2}_{i=0}$为$t$的三条边。$\{len_i\}^{2}_{i=0}$代表三边的边长。$t$的每条边会被均匀分成$\lceil len_i/l \rceil$段。$p_{ij}$表示三角形第i条边的第j个分割点。

算法流程:
\begin{enumerate}
    \item 找出三角面片最小的内角$\alpha$,不妨假定角$\alpha$的两条边为$e_0$, $e_1$。
    \item 将$e_0$、$e_1$分别均匀分成$\lceil len_0/l \rceil$、$\lceil len_0/l \rceil$段,产生的切割点的有序集合\footnote{切割点包括边的首尾端点}为$\{p_{0j}\}^{\lceil len_0/l \rceil}_{j=0}$、$\{p_{1j}\}^{\lceil len_1/l \rceil}_{j=0}$,且$j$沿角$\alpha$的顶点至另一端点方向依次增长。分割点如\autoref{subfig:clip1}所示。
    \item 将子三角形$p_{00}p_{01}p_{11}$分割下来,如\autoref{subfig:clip2}所示。
    \item 剩下的部分是一个类似于梯形的四边形,如\autoref{subfig:clip2}中红色部分所示。如果$\lceil len_0/l \rceil == \lceil len_1/l \rceil$,我们依次将四边形$\{p_{0j}p_{1j}p_{1(j+1)}p_{0(j+1)}\}^{\lceil len_0/l \rceil - 1}_{j=0}$分割成子三角形,如\autoref{subfig:clip6}、\autoref{subfig:clip9}所示,直到将所有四边形均分割完毕,然后直接跳转到步骤\ref{CVT};否则\footnote{$\lceil len_0/l \rceil \ne \lceil len_1/l \rceil$},我们依次将四边形$\{p_{0j}p_{1j}p_{1(j+1)}p_{0(j+1)}\}^{min(\lceil len_0/l \rceil, \lceil len_1/l \rceil) - 2}_{j=0}$分割成子三角形,剩余部分如\autoref{subfig:clip16}所示。\label{recursion}
    \item \autoref{subfig:clip16}中剩余部分若逆时针旋转90度,我们可以发现剩余部分和\autoref{subfig:clip3}类似。所以我们递归的进行步骤\ref{recursion},直到三角形分割完毕。\label{recursion2}
    \item 对分割结果进行5次CVT优化,使子三角形边长更加接近$l$。\label{CVT}
\end{enumerate}

在步骤\ref{recursion2}中,递归到最后阶段,可能会出现步骤\ref{recursion}处理不了的特殊情况。但是特殊情况的类型不会很多,所以我们为每一种特殊情况指定了对应的分割方案。

\begin{figure}[htbp]
	\centering
	\begin{subfigure}[b]{.32\textwidth}
		\centering
		\includegraphics[width = \textwidth]{clip_figure0.png}
		\caption{初始三角面片}\label{subfig:clip0}
	\end{subfigure}
	\begin{subfigure}[b]{.32\textwidth}
		\centering
		\includegraphics[width = \textwidth]{clip_figure1.png}
		\caption{初始三角面片}\label{subfig:clip1}
	\end{subfigure}
	\begin{subfigure}[b]{.32\textwidth}
		\centering
		\includegraphics[width = \textwidth]{clip_figure2.png}
		\caption{分割最小角$\alpha$的两条边}\label{subfig:clip2}
	\end{subfigure}

	\begin{subfigure}[b]{.32\textwidth}
		\centering
		\includegraphics[width = \textwidth]{clip_figure3.png}
		\caption{切割第一个子三角形}\label{subfig:clip3}
	\end{subfigure}
	\begin{subfigure}[b]{.32\textwidth}
		\centering
		\includegraphics[width = \textwidth]{clip_figure4.png}
		\caption{第一个待分割层}\label{subfig:clip4}
	\end{subfigure}
	\begin{subfigure}[b]{.32\textwidth}
		\centering
		\includegraphics[width = \textwidth]{clip_figure5.png}
		\caption{分割待分割层的上下底边}\label{subfig:clip5}
	\end{subfigure}

	\begin{subfigure}[b]{.32\textwidth}
		\centering
		\includegraphics[width = \textwidth]{clip_figure6.png}
		\caption{分割待侵害层的上下底边}\label{subfig:clip6}
	\end{subfigure}
	\begin{subfigure}[b]{.32\textwidth}
		\centering
		\includegraphics[width = \textwidth]{clip_figure9.png}
		\caption{分割待侵害层的上下底边}\label{subfig:clip9}
	\end{subfigure}
	\begin{subfigure}[b]{.32\textwidth}
		\centering
		\includegraphics[width = \textwidth]{clip_figure16.png}
		\caption{继续侵害红色部分}\label{subfig:clip16}
	\end{subfigure}

	\begin{subfigure}[b]{.32\textwidth}
		\centering
		\includegraphics[width = \textwidth]{clip_figure26.png}
		\caption{对特殊情况进行处理}\label{subfig:clip26}
	\end{subfigure}
	\begin{subfigure}[b]{.32\textwidth}
		\centering
		\includegraphics[width = \textwidth]{clip_figure33.png}
		\caption{分割结果}\label{subfig:clip33}
	\end{subfigure}
	\begin{subfigure}[b]{.32\textwidth}
		\centering
		\includegraphics[width = \textwidth]{clip_figure34.png}
		\caption{对结果进行CVT优化}\label{subfig:clip34}
	\end{subfigure}
	\caption{均匀三角剖分算法}\label{fig:clip}
\end{figure}

CVT\cite{du1999}优化可以使分割更加均匀。且该过程可以迭代进行,使分割结果越来越均匀。\autoref{fig:CVT}展示了迭代过程,红色是优化之后的结果,灰色的是优化之前的。我们可以发现,第5次及之后的迭代收益不大。所以我们最终对结果进行5次CVT迭代优化。

\begin{figure}[htbp]
	\centering
	\begin{subfigure}[b]{.49\textwidth}
		\centering
		\includegraphics[width = \textwidth]{cvt_for_paper0.png}
		\caption{第1次CVT优化}
	\end{subfigure}
	\begin{subfigure}[b]{.49\textwidth}
		\centering
		\includegraphics[width = \textwidth]{cvt_for_paper1.png}
		\caption{第2次CVT优化}
	\end{subfigure}

	\begin{subfigure}[b]{.49\textwidth}
		\centering
		\includegraphics[width = \textwidth]{cvt_for_paper2.png}
		\caption{第3次CVT优化}
	\end{subfigure}
	\begin{subfigure}[b]{.49\textwidth}
		\centering
		\includegraphics[width = \textwidth]{cvt_for_paper3.png}
		\caption{第4次CVT优化}
	\end{subfigure}

	\begin{subfigure}[b]{.49\textwidth}
		\centering
		\includegraphics[width = \textwidth]{cvt_for_paper4.png}
		\caption{第5次CVT优化}
	\end{subfigure}
	\begin{subfigure}[b]{.49\textwidth}
		\centering
		\includegraphics[width = \textwidth]{cvt_for_paper5.png}
		\caption{1~10次CVT优化}
	\end{subfigure}
    \caption{CVT优化结果} \label{fig:CVT}
\end{figure}

\section{关于$l$的讨论}
前文提到的$l$是一个很关键的参数,$l$不仅会影响算法效率、子三角形的质量\footnote{这里的质量表示接近正三角形的程度}、还能影响变形结果的精度。但是,$l$并不是影响这些的直接因素,对子三角形的质量和算法效率产生直接影响的是子三角形的数目,而对变形结果的精度产生直接影响的是子三角形的平均面积。$l$则通过改变子三角形数目或平均面积,间接地对算法效率、子三角形的质量、变形结果的精度产生影响。

定性的来说$l$越小,子三角形数目越多,面积越小,同时子三角形的质量越高,变形结果的误差越小,但算法的计算代价越高。为了进一步明确$l$对算法的影响,我们进行了一系列实验以定量的分析$l$。

实验选用了立方体模型,有12个面片,且模型被归一化到$[-1, 1]^3$。之所以选择由大三角面片组成的模型,是为了让三角面片随$l$变化切割出不同大小的子三角形。\autoref{fig:l-number}显示了$l$与子三角形数量的关系,蓝色表示以立方体为模型进行切割,立方体原始的三角面片平均面积为2.0,绿色表示以兔子玩偶为模型进行切割,兔子玩偶原始的三角面片平均面积为0.0007。从中我们可以发现兔子玩偶模型原始三角片面过小,当$l$大于某个值的时候,其不再被切割,子三角形的数目也保持不变,这样不利于我们探索$l$在对变形算法产生的影响。而立方体由于原始三角面片较大,能随$l$变大切割出不同数目的子三角形。可以帮助我们更好的研究$l$对算法的影响。

变形空间的次数是$3\times3\times3$,控制顶点个数是$5\times5\times5$。实验的自变量是$l$,由$0.1$增长到$2\sqrt{3}$\footnote{$2\sqrt{3}$为变形空间对角线长度}。因变量是算法效率、子三角形的质量、变形结果的精度。以下是实验结果。

%兔子模型的平均三角形面积为总表面积/三角形数目
%5.880918 / 8400 = 0.0007001092857142858
\begin{figure}[htbp]
	\centering
	\includegraphics[width = 0.8\textwidth]{l-number.png}
	\caption{$l$取值-子三角形数目关系图}\label{fig:l-number}
\end{figure}

\subsection{算法效率}
从\autoref{fig:l-number}可以看出不同的$l$可能产生相同的分割结果,所以\autoref{fig:l-time0}中,$l$与算法运算时间的关系不是很直观。如前文所述,$l$通过影响子三角形数目,从而影响算法运算时间,所以我们直接探究子三角形数目和算法运算时间的关系。如\autoref{fig:l-time1}中所示,大体上来看,算法运行时间与子三角形数目呈线性正相关。

\begin{figure}[htbp]
	\centering
	\begin{subfigure}[b]{.45\textwidth}
	    \centering
	    \includegraphics[width =\textwidth]{l-time0.png}
	    \caption{$l$取值-算法运行时间关系图}\label{fig:l-time0}
	\end{subfigure}
	\begin{subfigure}[b]{.45\textwidth}
	    \centering
	    \includegraphics[width =\textwidth]{l-time1.png}
	    \caption{三角形数目-算法运行时间关系图}\label{fig:l-time1}
	\end{subfigure}
	\caption{$l$取值-算法效率}\label{fig:l-time}
\end{figure}

\subsection{子三角形质量}
有很多指标都可以衡量三角形质量\cite{pebay2003},我们选取了其中计算量较少的一个$q(t)=r_t/max(\{e_i\}^{2}_{i=0})$,即三角形外接圆的半径除以其最长边。该指标的取值范围是$(0, \frac{\sqrt{3}}{2}]$,当三角形$t$为正三角形时,$q(t)$取到最大值$\frac{\sqrt{3}}{2}$;三角形$t$越狭长,$q(t)$的值越接近0。为了更直观的用该指标表示三角形质量,我们修改将$q(t)$修改为$q(t)=r_t/max(\{e_i\}^{2}_{i=0})*2\sqrt{3}$,使其取值范围为$(0, 1]$。\autoref{fig:triangle_quality_compare}可视化了两种三角形分割方法产生的三角形的质量,可见本文方法产生的三角形质量要好很多。

\begin{figure}[htbp]
	\centering
	\begin{subfigure}[b]{.45\textwidth}
		\centering
		\includegraphics[width = \textwidth]{clip_compare0.png}
		\caption{沿节点盒切割}\label{subfig:clip_compare0}
	\end{subfigure}%
	\begin{subfigure}[b]{.45\textwidth}
		\centering
		\includegraphics[width = \textwidth]{clip_compare1.png}
		\caption{本文方法}\label{subfig:clip_compare1}
	\end{subfigure}
	\caption{两种切割算法产生的子三角形的质量对比}\label{fig:triangle_quality_compare}
\end{figure}

对于不同的$l$对三角形质量的影响如\autoref{fig:l-quality}所示。我们分别作了$l$取值、子三角形数目与子三角形质量的关系图。可见$l$越小,三角形质量越高。但是由于切割算法中存在取整操作,所以\autoref{fig:l-quality}存在许多跳变的地方。

\begin{figure}[htbp]
	\centering
	\begin{subfigure}[b]{.45\textwidth}
		\centering
		\includegraphics[width = \textwidth]{l-quality0.png}
		\caption{$l$取值-子三角形质量关系图}\label{subfig:l-quality0}
	\end{subfigure}%
	\begin{subfigure}[b]{.45\textwidth}
		\centering
		\includegraphics[width = \textwidth]{l-quality1.png}
		\caption{子三角形数目-子三角形质量关系图}\label{subfig:l-quality1}
	\end{subfigure}
	\caption{$l$取值-子三角形的质量对比}\label{fig:l-quality}
\end{figure}

\subsection{变形结果精度}
相比于光滑自由变形,本文方法会产生跨节点盒的子三角形。这些三角形会引入额外的误差。但是只要子三角足够小,误差就能控制在可接受范围内。我们同样做了实验以验证$l$取值与变形误差的关系。从\autoref{fig:l-error}中可以看出,变形结果的几何误差基本与子三角形面积呈线性正相关。
\begin{figure}[htbp]
	\centering
	\begin{subfigure}[b]{.45\textwidth}
		\centering
		\includegraphics[width = \textwidth]{l-error0.png}
		\caption{$l$取值-子三角形质量关系图}\label{subfig:l-error0}
	\end{subfigure}%
	\begin{subfigure}[b]{.45\textwidth}
		\centering
		\includegraphics[width = \textwidth]{l-error1.png}
		\caption{子三角形数目-子三角形质量关系图}\label{subfig:l-error1}
	\end{subfigure}
	\caption{$l$取值-变形误差}\label{fig:l-error}
\end{figure}
