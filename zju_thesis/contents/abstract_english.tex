% !TEX root = ../main.tex

% 定义英文摘要和关键字

\begin{eabstract}
    Space deformation is one of key methods of shape manipulation and soft object animation in geometric modeling and computer graphics. Free-form Deformation (FFD) is the representative one among them. Due to its simplicity, intuitiveness, flexibility, it has been integrated into mainstream commercial animation softwares, such as 3DS Max, Maya, Softimage XSI, etc. In this thesis, an improved smooth FFD is proposed to address the problems of hardware-dependence and robustness in the traditional GPU-based FFD methods.

    Firstly, we propose a new method of uniformly subdividing the original model in the framework of smooth FFD. It receives a triangle as input and subdivides it to sub-triangles whose side length and area will be as equal as possible. The method can avoid pathological cases in traditional GPU-based FFD and generate the sub-triangles as uniform as possible. Furthermore, the subdivision method can be fully implemented on GPU. As a result, the smooth FFD becomes more robust and efficient.

    To overcome the problem of hardware-dependence in previous GPU-based FFD, we also proposed a new GPU-based improved smooth FFD by using OpenGL compute shader. All of the intensive computations are designed and implemented on GPU provided that the universality and efficiency of the proposed algorithm are guaranteed.  We also integrate the proposed algorithm into a 3D geometric modeling APP, namely, Wowtao, by the aid of the cross-platform property of OpenGL. It verifies the universality and feasibility of the proposed algorithm to achieve 3D shape editing in a resource-constrained hardware.

Finally, we compare the proposed method with the smooth FFD and accurate FFD thoroughly and show that the proposed method has the advantages of efficiency, robustness and cross-platform. It extends the potential application scenarios of FFD.
\end{eabstract}

\ekeywords{smooth FFD, sharp feature preserving, GPU, OpenGL compute shader, normal field}
