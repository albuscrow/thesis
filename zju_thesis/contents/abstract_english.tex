% !TEX root = ../main.tex

% 定义英文摘要和关键字

\begin{eabstract}
    Free-form deformation\cite{Sederberg86}(FFD) is a prevalent shape manipulation and shape animation method in computer graphics and geometric modeling. It is widely used in Computer-aided Design, Computer Animation, Medical Imaging and so on. On the other hand, FFD is getting more and more attention from industrial and academic since Sederberg and Parry proposed it because of its scalability. FFD is not only a specific method for deformation, but also a framework of editing 3D models.

    On the basis of this framework, many researchers have proposed a lot of variants of FFD to improve its limitation, such as the aliasing of deformation result, the cumbersome interaction, low efficiency and so on.

    Cui's Smooth free-form deformation\cite{Cui15} is also a variant of FFD. In smooth free-form deformation\cite{Cui15} of a polygonal object, the linear geometry, e.g., triangles or planar polygons will be subdivided against a knot box, but it will produce pathological triangles which will contribute to many problem about efficiency and robustness. On the other hand, cui utilize the parallel computing power of the GPU via CUDA, which is limited to NVIDIA hardware. Thus, his method can not be used on neithor AMD GPU nor mobile platform. In this paper, we propose an approach which will be used to clip the original model to avoid pathological case. Moreover, we implement the whole framework of smooth FFD with OpenGL compute shader in order to make it more general. Our subdivision method will produce sub-triangles as regular as possible, but cannot guarantee that each sub-triangle lies in a knot box. Accordingly the method will introduce acceptable errors. However, the trade-off is worthwhile in accordance with Error analysis. We also extend the deformation framework to the mobile phone by the aid of the across-platform of OpenGL.
\end{eabstract}

\ekeywords{smooth FFD, sharp feature preserving, GPU, OpenGL compute shader, normal field}
