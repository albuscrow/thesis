% !TEX root = ../main.tex

\chapter{三角贝赛尔曲面片表示的光滑自由变形概述}
本文方法作为光滑自由变形\cite{Cui15}的改进算法,依旧遵循经典自由变形的算法框架。下文将沿着该框架的流程,对光滑自由变形算法进行简单介绍。

\section{定义变形空间}
精确自由变形选用B样条体作为变形空间,记作$\mathbf R(u,v,w)$:
\begin{equation}
	\footnotesize
	{\mathbf R}(u,v,w) 
	= \sum_{i=0}^{m_u-1}\sum_{j=0}^{m_v-1}\sum_{k=0}^{m_w-1} {\mathbf
	R}_{ijk}N_{i,n_u}(u)N_{j,n_v}(v)N_{k,n_w}(w)
	\label{equ:Ruvw}
\end{equation}
其中,$n_u$、$n_v$、$n_w$与$m_u$、$m_v$、$m_w$分别表示B样条体三个维度上的次数与控制顶点个数。$\{\mathbf R_{ijk}\}_{i=0,\hspace{6 pt} j=0,\hspace{8 pt} k=0}^{m_u-1,m_v-1,m_w-1}$表示$m_u\times m_v\times m_w$个控制顶点。$\{N_{i,n_u}(u)\}_{i=0}^{m_u-1}$, $\{N_{j,n_v}(v)\}_{j=0}^{m_v-1}$ 和 $\{N_{k,n_w}(w)\}_{k=0}^{m_w-1}$是B样条基函数。

此B样条体三个维度上的节点向量分别是$\{u_i\}^{n_u+m_u}_{i=0}$, $\{v_i\}^{n_v+m_v}_{j=0}$ 和 $\{w_k\}^{n_k+m_k}_{k=0}$。由此定义的三维区域$[u_i, u_{i+1}] \times [v_j, v_{j+1}] \times [w_k, w_{k+1}]$我们称之为节点盒,其中$n_u\leq i < m_u$,$n_v\leq j < m_v$,$n_w\leq k < m_w$。

\subsection{沿节点盒切割多边形面片}
根据论文\cite{Feng98, Feng00},完全在某个节点盒内的三角面片,其变形后的结果是一个三角贝赛尔曲面片,记作${\mathbf P}(u,v,w)$:\label{section:split}

\begin{equation}
	\footnotesize
	{\mathbf P}(u,v,w)
	= \sum_{\substack{i+j+k=n \\ 0\leq i,j,k\leq n}} {\mathbf P}_{ijk}B^n_{ijk}(u,v,w), \hspace{8 pt} u,v,w\ge0,
		\hspace{8 pt}u+v+w=1
	\label{equ:Puvw}
\end{equation}

其中$\{B_{ijk}^n(u,v,w)=\frac{n!}{i!j!k!}u^iv^jw^k \mid i+j+k=n\}$是定义在一个三角形上的伯恩斯坦基函数,$n(=n_u+n_v+n_w)$为曲面片的次数,$\{\mathbf P_{ijk} \mid i+j+k=n\}$为些三角贝赛尔曲面片的$m(=(n+1)(n+2)/2)$个控制顶点。

所以由平面多边形构成的模型在嵌入B样条体之后,需要沿上述节点盒切割,且得到的非三角形的子多边形还需再进行三角化。以保证变形后的结果可以用三角贝赛尔曲面片表示。

\section{模型嵌入变形空间}
“嵌入”过程,其实是计算“待变形的对象”在变形空间中的参数坐标的过程。从用户角度来看,“侍变形对角”就是待变形的模型,但是从算法实现的角度来看,真正参与变形的实际上是模型上的采样点。因此,“嵌入”就是通过嵌入函数$U=E(X)$将采样点从笛卡尔坐标系映射到变形中间中的过程。其中$X$为采样点在笛卡尔坐标系中的坐标,$U$为采样点在变形空间的参数坐标。

所以嵌入变形空间这一过程有两个要点:确定变形函数、选取采样点。

\subsection{确定变形函数}

嵌入函数$E(X)$由变形空间决定。光滑自由变形通过\cite{Feng02}中的方法构造B样条体,使得其具有如下性质:嵌入其中的点的参数坐标与其笛卡尔坐标系的坐标相等。即$E(X)=X$。

\subsection{采样点的选取}
不同的种类的FFD算法,需要选取不同的采样点。

传统自由变形以待变形模型的顶点为采样点,这将导致如图\autoref{fig:sample_problem}所示的走样问题。均匀加密采样只能在一定程度上解决走样问题,且随着采样点密度的增加,无论是时间还是空间上的开销均会显著增加。自适应的加密采样是在加密采样思路下解决走样问题的更进一步的尝试。相对均匀加密采样而言,其虽然能减少计算量,但是实现复杂,无法很好的应对一些特殊情况。以上两种方法匀无法人根本上解决走样问题。

精确自由变形\cite{Feng00}通过另一种思路从根本上解决了走样问题。如\ref{section:split}所述,三角面片变形后是一个三角贝赛尔曲面片,所以,只要用FFD求得三角面片上的$m$个均匀采样点在变形之后的位置,就可以通过多项式插值高效计算出三角贝赛尔曲面片的控制顶点。这一方法相对于加密采样的优势在于其结果是一个解析表达的贝赛尔曲面片,可以根据需求细化绘制出不同精度的变形结果,而无需改变采样点个数。

光滑自由变形\cite{Cui15}进一步优化了精确自由变形的计算量。首先,Cui指出作为变形结果的三角贝赛尔曲面片,其次数往往高于三次,高次的结果不仅会直接产生高昂的计算代价,还会间接的增加算法其它部分的复杂度。然后,Cui通过实验发现,三次的三角贝赛尔曲面片能提供足够的灵活性以拟合精确的自由变形结果。所以,Cui选用了三次贝赛尔曲面片来表示变形结果。该曲面片由带约束的拟合方法求出,该方法需要$3\times3$个约束点,$m$个拟合点。所以,光滑自由变形共需要$9+m$个采样点。\autoref{fig:saffd_sample_point}中展示了当$m=4$时,采样点的选取方式,黄色点为约束点,蓝色点为拟合点。

\begin{figure}[htbp]
	\centering
	\includegraphics[width = 0.5\linewidth]{saffd_sample_point.png}
	\caption{三角面片上的拟合点的约束点示意图}\label{fig:saffd_sample_point}
\end{figure}


\section{变形}
如前文所述,实际参与变形的是控制顶点。通过上述步骤,我们已经将采样点嵌入到B样条体中。当用户编辑B样条体控制顶点的位置后,算法先将采样点的参数坐标和控制顶点位置代入\autoref{equ:Ruvw},就可以求得采样点新的位置。再以这些采样点新的位置为输入,通过带约束的拟合方法,求出作为变形结果的三角贝赛尔曲面的控制顶点。

为了得到视觉上更加细腻的变形结果,算法还将根据法向信息对控制顶点进行微调。最终输出光滑并保持尖锐特征的结果。

以上就是光滑自由变形的大致步骤,下面将介绍本文算法在此基础上进行的改进。
